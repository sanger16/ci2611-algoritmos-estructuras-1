\documentclass[hidelinks]{article}
%Packages
\usepackage{amsmath}
\usepackage{amssymb}
%\usepackage{mathabx}
\usepackage{dirtytalk}
\usepackage{textcomp}
\usepackage[margin=1in]{geometry}
\usepackage{tikz}
\usetikzlibrary{arrows,positioning,shapes,fit,calc}
\pgfdeclarelayer{background}
\pgfsetlayers{background,main}
\usepackage{hyperref}
\usepackage{setspace}
\usepackage{framed}
\usepackage{fancyhdr}
\pagestyle{fancy}
\setlength{\headheight}{13.07225pt}
\rfoot[CI2611 - Problemario]{CI2611 - Problemario - P3}
%\rfoot{\thepage}
\doublespacing
\usepackage{needspace}
\usepackage{array}
\usepackage{booktabs}
\usepackage{indentfirst}
%---
%Customize
\newcommand{\myparagraph}[1]{\paragraph{#1}\mbox{}\\}
\newcommand\myrule[1]{\multicolumn{1}{| l}{#1}}
\def\multiset#1#2{\ensuremath{\left(\kern-.3em\left(\genfrac{}{}{0pt}{}{#1}{#2}\right)\kern-.3em\right)}}
\renewcommand{\contentsname}{Contenido}
%---

%--- No page break
\newenvironment{absolutelynopagebreak}
{\Needspace{10\baselineskip}\begin{quote}}
		{\end{quote}}
%---

\title{CI2611 - Algoritmos y estructuras I \\ Parcial 3}
\date{Abr-Jul 2025}
\author{Daniel Delgado}
\begin{document}
\maketitle

\tableofcontents

\newpage

\section{Resumen Parcial 3}

\subsection{Recursión}\par

\subsubsection{Procedimientos y Funciones Recursivas}

\begin{absolutelynopagebreak}
	$\textbf{proc} \; p \; (in \; x; \; in-out \; y; \; out \; z)$ \par
	$\quad \{P_{def}\}$ \par
	$\quad \{Q_{def}\}$\par
	$\quad \{Cota: t\}$\par
	$\qquad [$\par
			$\qquad \quad \dots$\par
			$\qquad \quad \{P_{llam}\} \; p(E,a,b) \; \{Q_{llam}\}$\par
			$\qquad  \quad \dots $\par
			$\qquad ]$\par
\end{absolutelynopagebreak}

\begin{absolutelynopagebreak}
	$\textbf{func} \; f \; (x : T) \rightarrow T^{\prime}$ \par
	$\quad \{P_{def}\}$ \par
	$\quad \{Q_{def}\}$\par
	$\quad \{Cota: t\}$\par
	$\qquad [\langle$ Cuerpo que debe tener una llamada a f $\rangle$\par
			$\qquad  \quad >> E$\par
			$\qquad ]$\par
\end{absolutelynopagebreak}

\subsubsection{Correctitud de Recursividad}

\textbf{Prueba 1}\par
Se debe demostrar la siguiente tripleta, igualando la función de cota a una
constante. Esta demostración implica aplicar las técnicas que se vieron para el
primer parcial de acuerdo al cuerpo del procedimiento.\par
$\{P_{def} \land t = X\}$\par
$\langle$ Cuerpo del procedimiento $\rangle$\par
$\{Q_{def}\}$\par

\textbf{Prueba 2}\par
Demostración de la disminución de la cota en cada llamada recursiva en la
primera prueba.\par
$2.1. [P_{llam} \Rightarrow (P_{def} \land t < X)(x,y := E, a)]$\par
$2.2. [P_{llam}(a,b := A,B) \land Q_{def}(x,y_0,y,z := E(a,b := A,B), A, a, b) \Rightarrow Q_{llam}]$\par

\textbf{Prueba 3}\par
Demostración de función acotada.\par
$[P_{def} \Rightarrow t \geq 0]$\par

\subsection{Invariante de Cola}\par

\subsubsection{Caso 1}

Se tiene una función $F$ de la siguiente forma,\par

\begin{equation}
	F(X) =
	\begin{cases}
		h_1(X)    & \text{si } b_1(X)                                                               \\
		h_2(X)    & \text{si } b_2(X)                                                               \\
		\dots                                                                                       \\
		h_m(X)    & \text{si } b_m(X)                                                               \\
		F(g_1(X)) & \text{si } \neg b_1(X) \land \neg b_2(X) \dots \neg b_m(X) \land c(X)           \\
		F(g_2(X)) & \text{si } \neg \neg b_1(X) \land \neg b_2(X) \dots \neg b_m(X) \land \neg c(X)
	\end{cases}
\end{equation} \par
Donde,\par

$X = x_1,x_2,\dots,x_k$ con $k \in \mathbb{N}$\par
$h_i$ con $i \in \mathbb{N}$ corresponde a los casos bases definidos por $F$.\par
$F(g_j(X))$ con $j \in \mathbb{N}$ corresponde a los casos recursivos definidos por $F$.

Se quiere derivar un programa cuya postcondición es $r = F(\overline{A})$ para
algún $\overline{A} = (a_1,a_2, \dots, a_k)$. Se define el Invariante de cola
como.

\begin{center}
	Inv: $F(X) = F(\overline{A})$
\end{center}\par

El programa derivado tiene la forma siguiente.\par

\begin{absolutelynopagebreak}
	[\par
		$\quad const \; x_1,x_2,\dots,x_k: T;$\par
		$\quad x := a_1,a_2, \dots, a_k$\par
		$\quad \{Inv: F(x_1,x_2,\dots,x_k) = F(a_1,a_2, \dots, a_k)\}$\par
		$\qquad \quad do \; \neg b_1(x_1,x_2,\dots,x_k) \land \neg b_2(x_1,x_2,\dots,x_k) \dots \neg b_m(x_1,x_2,\dots,x_k) \rightarrow$ \par
		$\qquad \qquad if \;  c(x_1,x_2,\dots,x_k) \rightarrow$ \par
		$\qquad \qquad \quad x := g_1(x_1,x_2,\dots,x_k)$ \par
		$\qquad \qquad [\;] \; \neg c(x_1,x_2,\dots,x_k) \rightarrow$ \par
		$\qquad \qquad \quad x := g_2(x_1,x_2,\dots,x_k)$ \par
		$\qquad \qquad fi$ \par
		$\qquad \quad od$ \par
		$\qquad \quad if \; b_1(x_1,x_2,\dots,x_k) \land \neg b_2(x_1,x_2,\dots,x_k) \dots \neg b_m(x_1,x_2,\dots,x_k) \rightarrow$ \par
		$\qquad \qquad r = h_1(x_1,x_2,\dots,x_k)$\par
		$\qquad \quad [\;] \; \neg b_1(x_1,x_2,\dots,x_k) \land b_2(x_1,x_2,\dots,x_k) \dots \neg b_m(x_1,x_2,\dots,x_k) \rightarrow$ \par
		$\qquad \qquad r = h_2(x_1,x_2,\dots,x_k)$\par
		$\qquad \quad \dots$ \par
		$\qquad \quad [\;] \; \neg b_1(x_1,x_2,\dots,x_k) \land \neg b_2(x_1,x_2,\dots,x_k) \dots b_m(x_1,x_2,\dots,x_k) \rightarrow$ \par
		$\qquad \qquad r = h_m(x_1,x_2,\dots,x_k)$\par
		$\qquad \quad [\;] \; \neg b_1(x_1,x_2,\dots,x_k) \land \neg b_2(x_1,x_2,\dots,x_k) \dots \neg b_m(x_1,x_2,\dots,x_k) \rightarrow$ \par
		$\qquad \qquad skip$\par
		$\qquad \quad fi$ \par
		$\quad \{r =F(a_1,a_2, \dots, a_k)\}$ \par
	]\par
\end{absolutelynopagebreak}\par

Para el caso particular de un solo caso base, una sola variable y un solo caso
recursivo.\par

\begin{equation}
	F(x) =
	\begin{cases}
		h(x)    & \text{si } b(x)      \\
		F(g(x)) & \text{si } \neg b(x) \\
	\end{cases}
\end{equation} \par

Donde $h(x)$ es fácil de calcular en una sola iteración.\par

El invariante está definido por la expresión siguiente, con $A$ un valor
cualquiera en donde $F$ está definido y represemta el valor de x para el cual
se desea obtener un resultado del programa.\par

\begin{center}
	Inv: $F(x) = F(A)$
\end{center}

De donde se deriva el siguiente esquema de programa.\par

\begin{absolutelynopagebreak}
	[\par
		$\quad const \; x: T;$\par
		$\quad x := A$\par
		$\quad \{Inv: F(x) = F(A)\}$\par
		$\qquad \quad do \; \neg b(x) \rightarrow$ \par
		$\qquad \qquad x := g(x)$ \par
		$\qquad \quad od$ \par
		$\qquad \quad r = h(x)$ \par
		$\quad \{r =F(A)\}$ \par
	]\par
\end{absolutelynopagebreak}\par

\subsubsection{Caso 2}

Se tiene una función $F$ de la siguiente forma.\par

\begin{equation}
	F(x) =
	\begin{cases}
		h(x)                 & \text{si } b(x)      \\
		F(g(x)) \otimes j(x) & \text{si } \neg b(x) \\
	\end{cases}
\end{equation} \par

Donde $\otimes$ es una operación \textbf{asociativa} con \textbf{elemento
	neutro $e$}.\par

Se quiere derivar un programa cuya postcondición es $r = F(A)$ para algún $A$
en el dominio de $F$.\par

El invariante de cola se define como sigue.\par

\textit{"Resultado Final" = "Lo que lleva" $\otimes$ "Lo que falta por calcular"}.\par

Lo cual se traduce en la expresión matemática siguiente.\par

\begin{center}
	Inv: $F(A) = r \otimes F(x)$
\end{center}

\begin{enumerate}
	\item Se inicializan las variables $r$ y $x$.\par
	      $r, x := e, A$\par
	      Esto garantiza que el invariante se cumple antes de entrar al bucle.

	\item Si $b(x)$ se cumple entonces $F(x) = h(x)$ entonces se cumple.\par
	      $F(A) = r \otimes F(x) \equiv F(A) = r \otimes h(x)$\par
	      Esto garantiza que el invariante se cumple al finalizar el bucle.\par

	\item Si $b(x)$ no se cumple entonces $F(x) = F(g(x)) \otimes j(x)$, por lo tanto se
	      tiene lo siguiente.\par
	      $\qquad F(A) = r \otimes F(x) \equiv r \otimes (F(g(x)) \otimes j(x)) \equiv (r \otimes j(x)) \otimes F(g(x))$\par
	      $\equiv \qquad \langle F(x) \rangle$\par
	      $\qquad r \otimes (F(g(x)) \otimes j(x))$\par
	      $\equiv \qquad \langle Asociatividad \rangle$\par
	      $\qquad (r \otimes j(x)) \otimes F(g(x))$\par
	      El invariante se cumple siempre que se hagan las siguientes asignaciones dentro
	      del bucle.\par
	      \begin{center}
		      $r := r \otimes j(x)$\par
		      $x := g(x)$\par
	      \end{center}
	      De esta manera el invariante se cumple durante el bucle.\par
\end{enumerate}

Finalmente, el esquema del programa quedará.\par

\begin{absolutelynopagebreak}
	[\par
		$\quad const \; x: T;$\par
		$\quad x, r := A, e;$\par
		$\quad \{Inv: F(A) = r \otimes F(x)\}$\par
		$\qquad \quad do \; \neg b(x) \rightarrow$ \par
		$\qquad \qquad r, x := r \otimes j(x), g(x)$ \par
		$\qquad \quad od$ \par
		$\quad \{ F(A) = r \otimes a\}$ \par
		$\qquad \quad r = h(x)$ \par
		$\quad \{r =F(A)\}$ \par
	]\par
\end{absolutelynopagebreak}\par

\subsection{Recursión de Cola}\par
Regularmente se busca transformar una función recursiva en una función con
recursión de cola, para ello es necesario que la última llamada recursiva sea
la última instrucción en la función, de manera que no hay operaciones
pendientes después de la llamada recursiva.\par

Esquema general para el análisis.\par

\begin{enumerate}
	\item \textbf{Identificar} las expresiones de la recursión original que hacen que no sea de cola.
	\item \textbf{Sustituir} las llamadas recursivas por acumuladores, los cuales mantendrán actualizados los cálculos realizados en iteraciones pasadas. Estos acumuladores será uno por cada llamada recursiva y serán parámetros de la función. Se deberá indicar una expresión matemática que de forma general permita determinar el valor del acumulador: al inicio, al final y en una actualización del iterador.
	\item \textbf{Agregar o Identificar (si ya existe)} en los parámetros un índice que permita determinar en qué iteración nos encontramos. Puede ser creciente o decreciente.
	\item \textbf{Transformar} la llamada recursiva en una de cola de tal manera que el iterador disminuya o aumente en uno y el acumulador se actualice debidamente.
	\item Identificar la expresión que representa el caso base la cual debe estar escrita
	      en función del acumulador.
	\item Identificar el valor inicial del acumulador de tal manera que el resultado sea
	      igual al de la recursión normal.
\end{enumerate}

\newpage
\vspace*{\fill}
\hspace*{\fill} --- PROBLEMAS RESUELTOS --- \hspace*{\fill}
\vspace*{\fill}
\thispagestyle{empty}
\newpage

\section{Recursión}

\subsection{Ejercicios de clase}\par

\subsubsection{Practica Ene-Mar 2025}

Demuestre el siguiente procedimiento recursivo.\par

\begin{absolutelynopagebreak}
	$\textbf{proc} \; productoria(entrada \; i, N: int; A: \; array[0..N) \; of \; int; salida \; r: int)$ \par
	$\quad \{Pre: 0 \leq i \leq N\}$;\par
	$\quad \{Post: r = (\prod k | i <= k < N: A[k])\}$;\par
	$\quad \{Cota: N - i\}$;\par
	$\qquad [$\par
			$\qquad \quad if \; i = N \rightarrow r := 1$\par
			$\qquad \quad [\;] \; i < N \rightarrow $\par
			$\qquad \qquad productoria(i + 1, N, A, r)$;\par
			$\qquad \qquad r := A[i] * r $\par
			$\qquad \quad fi$\par
			$\qquad ]$\par
\end{absolutelynopagebreak}\par

\textbf{Solución}

\begin{itemize}
	\item Prueba 1.\par
	      Se debe probar la siguiente tripleta.\par
	      $\{P_{def} \land t = X\} \; S \; \{Q_{def}\}$\par
	      Es decir,\par
	      \begin{absolutelynopagebreak}
		      $\{0 \leq i \leq N \land N - i = X\}$\par
		      $\quad  if \; i = N \rightarrow r := 1$\par
		      $\quad  [\;] \; i < N \rightarrow $\par
		      $\quad \qquad productoria(i + 1, N, A, r)$;\par
		      $\quad \qquad r := A[i] * r $\par
		      $\quad  fi$\par
		      $\{r = (\prod k | i <= k < N: A[k])\}$\par
	      \end{absolutelynopagebreak}

	      Se debe probar la selección, con las siguientes pruebas.\par
	      $1.1 \; P \Rightarrow B_0 \lor B_1$\par
	      $1.2.0 \; \{P \land B_0\} S_0 \{Q\}$\par
	      $1.2.1 \; \{P \land B_1\} S_1 \{Q\}$\par

	      \textbf{Prueba 1.1.}\par
	      $0 \leq i \leq N \land N - i = X \Rightarrow i = N \lor i < N$\par

	      Suposición del antecedente, empezando por true.\par

	      Suponemos:\par
	      $H_0: \; 0 \leq i \leq N$\par
	      $H_1: \; N - i = X$\par

	      $\qquad true$\par
	      $\equiv \qquad \langle H0 \rangle$\par
	      $\qquad 0 \leq i \leq N$\par
	      $\equiv \qquad \langle a \leq b \leq c \equiv a \leq b \land b \leq c \rangle$\par
	      $\qquad 0 \leq i \land i \leq N$\par
	      $\equiv \qquad \langle p \land q \Rightarrow p \rangle$\par
	      $\qquad i \leq N$\par
	      $\equiv \qquad \langle$ Aritmética $\rangle$\par
	      $\qquad i = N \lor i < N$\par
	      $\blacksquare$\par

	      \textbf{Prueba 1.2.0}\par
	      Se debe probar la siguiente tripleta.\par
	      $\{0 \leq i \leq N \land N - i = X \land i = N\} \; r := 1 \; \{r = (\prod k | i <= k < N: A[k])\} $\par
	      Por regla de la asignación, debemos probar el siguiente predicado.\par
	      $0 \leq i \leq N \land N - i = X \land i = N\ \Rightarrow (r = (\prod k | i <= k < N: A[k]))(r := 1) $\par
	      Por suposición del antecedente, empezando por el consecuente para alcanzar
	      true.\par
	      Suponemos:\par
	      $H_0: \; 0 \leq i \leq N$\par
	      $H_1: \; N - i = X N$\par
	      $H_2: \; i = N$\par
	      $\qquad (r = (\prod k | i <= k < N: A[k]))(r := 1)$\par
	      $\equiv \qquad \langle$ Sustitucion Textual $\rangle$\par
	      $\qquad 1 = (\prod k | i <= k < N: A[k])$\par
	      $\equiv \qquad \langle$ Sustitución por $H2\rangle$\par
	      $\qquad 1 = (\prod k | N <= k < N: A[k])$\par
	      $\equiv \qquad \langle$ Rango vacío $(\prod k | false: x) = 1\rangle$\par
	      $\qquad 1 = 1$\par
	      $\equiv \qquad \langle$ Aritmética $\rangle$\par
	      $\qquad true$\par
	      $\blacksquare$\par

	      \textbf{Prueba 1.2.1}\par
	      Se debe probar la siguiente tripleta.\par
	      $\{0 \leq i \leq N \land N - i = X \land i < N\} \; productoria(i + 1, N, A, r);r := A[i] * r \; \{r = (\prod k | i <= k < N: A[k])\} $\par
	      Dado que se tiene una secuenciación se debe obtener la aserción intermedia,
	      para lo cual podemos usar el precondición más débil para la asignación.\par
	      $\qquad wp(r := A[i]*r).(r = (\prod k | i <= k < N: A[k]))$\par
	      $\equiv \qquad \langle$ Definición wp$\rangle$\par
	      $\qquad (r = (\prod k | i <= k < N: A[k]))(r := A[i]*r)$\par
	      $\equiv \qquad \langle$ Sustitución textual$\rangle$\par
	      $\qquad A[i]*r = (\prod k | i <= k < N: A[k])$\par

	      De esta manera tenemos las siguientes tripletas.\par
	      $\{0 \leq i \leq N \land N - i = X \land i < N\}$\par
	      $\quad productoria(i + 1, N, A, r);$\par
	      $\{A[i]*r = (\prod k | i <= k < N: A[k])\}$\par
	      $\quad r := A[i] * r$\par
	      $\{r = (\prod k | i <= k < N: A[k])\} $\par

	      Como la precondición más débil asegura correctitud, se considera que la
	      asigación ya está probada, por lo tanto, los queda probar la siguiente.\par

	      $\{0 \leq i \leq N \land N - i = X \land i < N\}$\par
	      $\quad productoria(i + 1, N, A, r);$\par
	      $\{A[i]*r = (\prod k | i <= k < N: A[k])\}$\par

		  Dado que es un procedimiento recursivo se debe probar.\par
		  $1. \; P_{llam} \Rightarrow (P_{def} \land t < X)(x,y := E,a)$\par
		  $2. \; P_{llam}(a,b := A,B) \land Q_{def}(x,y_0,y,z := E(a,b := A,B), A, a,b) \Rightarrow Q_{llam}$\par

		  \textbf{Prueba 1.2.1.1}\par
		  $0 \leq i \leq N \land N - i = X \land i < N \Rightarrow (0 \leq i \leq N \land N - i < X)(i,N, A := i + 1,N,A)$\par

		  Suposición del antecedente, empezando por el consecuente para alcanzar true.\par

		  Suponiendo:\par
		  $H_0: \; 0 \leq i \leq N$\par
		  $H_1: \; N - i = X$\par
		  $H_2: \; i < N$\par

		  $\qquad (0 \leq i \leq N \land N - i < X)(i,N, A := i + 1,N,A)$\par
	      $\equiv \qquad \langle$ Sustitución textual$\rangle$\par
		  $\qquad 0 \leq i + 1 \leq N \land N - (i + 1) < X$\par
	      $\equiv \qquad \langle$ Aritmética$\rangle$\par
		  $\qquad 0 \leq i + 1 \leq N \land N - i - 1 < X$\par
	      $\equiv \qquad \langle$ Sustitución por $H_1\rangle$\par
		  $\qquad 0 \leq i + 1 \leq N \land X - 1 < X$\par
	      $\equiv \qquad \langle$ $a - 1 < a \equiv true\rangle$\par
		  $\qquad 0 \leq i + 1 \leq N$\par
	      $\equiv \qquad \langle$ $a \leq b \leq c \equiv a \leq b \land b \leq c\rangle$\par
		  $\qquad 0 \leq i + 1 \land i + 1 \leq N$\par
	      $\equiv \qquad \langle$ Por $H_2: \; i < N \equiv true \Rightarrow i + 1 \leq N \rangle$\par
		  $\qquad 0 \leq i + 1$\par
	      $\equiv \qquad \langle$ Por $H_1: \; 0 \leq i \equiv true \Rightarrow 0 \leq i + 1 \rangle$\par
		  $\qquad true$\par
		  $\blacksquare$\par

		  \textbf{Prueba 1.2.1.2}\par

\end{itemize}

\newpage

\section{Invariante de Cola}

\subsection{Ejercicios Kaldewaij}\par

\subsubsection{Ejemplo}

Derivar programa para la exponenciación $x^y$.\par

\begin{absolutelynopagebreak}
	[\par
		$\quad const \; A, B: int;$\par
		$\quad var \; r : int;$\par
		$\qquad \quad \langle Exponenciancion \rangle$ \par
		$\quad \{r =A^B\}$ \par
	]\par
\end{absolutelynopagebreak}\par

\begin{enumerate}
	\item Se define la función $F$\par

	      \begin{equation}
		      F(x,y) =
		      \begin{cases}
			      1                                   & \text{si } y = 0                       \\
			      1 \cdot F(x \cdot x, y \; div \; 2) & \text{si } y \neq 0 \land y \mod 2 = 0 \\
			      x \cdot F(x, y - 1)                 & \text{si } y \neq 0 \land y \mod 2 = 1
		      \end{cases}
	      \end{equation} \par

	\item Se define el invariante: $P: F(A,B) = r \cdot F(x,y) \equiv A^B = r \cdot x^y$.\par
	\item Variables y valores iniciales.\par

	      EL operador principal de F es la multiplicación cuyo elemento neutro es 1, por
	      lo tanto nuestra variable de resultador $r$ se inicializa en 1, mientras que
	      $x$ se inicializa en su valor máximo $A$ y $y$ igualmente en su valor máximo B.\par

	      \begin{center}
		      $r,x,y : = 1, A, B$
	      \end{center}

	\item Para validar que el invariante se cumple en todo momento verificamos.

	      \begin{itemize}
		      \item Cuando $y \neq 0$\par

		            $\qquad F(A,B) = r \cdot F(x,y)$\par
		            $\equiv \qquad \langle F(x,y)$ en $ y \neq 0 \rangle$\par
		            $\qquad F(A,B) = r \cdot 1$\par
		            $\equiv \qquad \langle$ Operando $\rangle$\par
		            $\qquad F(A,B) = r$\par
		            $\equiv \qquad \langle$ Definición de $F\rangle$\par
		            $\qquad A^B = r$\par

		      \item Cuando $y = 0$.\par

		            \begin{itemize}
			            \item Cuando $y \mod 2 = 0$.\par
			                  $\qquad F(A,B) = r \cdot F(x,y)$\par
			                  $\equiv \qquad \langle $ Definición de $ F \rangle$\par
			                  $\qquad F(A,B) = r \cdot F(x \cdot x, y \; div \; 2)$\par
			                  $\equiv \qquad \langle P$  $\rangle$\par
			                  $\qquad F(A,B) = r \cdot (x \cdot x)^y$\par
			                  $\equiv \qquad \langle $ S.T sobre P $\rangle$\par
			                  $\qquad P(x,y := x \cdot x, y \; div \; 2)$\par
			                  Las actualizaciones son las siguientes.\par

			                  \begin{center}
				                  $x,y := x \cdot x, y \; div \; 2$\par
			                  \end{center}
			            \item Cuando $y \mod 2 = 1$.\par
			                  $\qquad F(A,B) = r \cdot F(x,y)$\par
			                  $\equiv \qquad \langle $ Definición de $ F \rangle$\par
			                  $\qquad F(A,B) = r \cdot x \cdot F(x, y - 1)$\par
			                  $\equiv \qquad \langle P$  $\rangle$\par
			                  $\qquad F(A,B) = r \cdot x \cdot (x)^{y-1}$\par
			                  $\equiv \qquad \langle $ S.T sobre P $\rangle$\par
			                  $\qquad P(r,y := r \cdot x, y - 1)$\par
			                  Las actualizaciones son las siguientes.\par

			                  \begin{center}
				                  $r,y := r \cdot x, y - 1$\par
			                  \end{center}
		            \end{itemize}
	      \end{itemize}

	      Finalmente, el programa es el siguiente.\par

	      \begin{absolutelynopagebreak}
		      [\par
			      $\quad const \; A, B: int;$\par
			      $\quad var \; r, x, y : int;$\par
			      $\quad r, x, y := 1, A, B;$\par
			      $\quad \{Inv:\; r \cdot x^y = A^B \land y \geq 0\}\{Cota: \; y\}$ \par
			      $\qquad do \; y \neq 0 \rightarrow$ \par
			      $\qquad \quad if \; y \mod 2 = 0 \rightarrow$ \par
			      $\qquad \quad \quad x, y := x \cdot x, y \; div \; 2$ \par
			      $\qquad \quad [\;] \; y \mod 2 = 1 \rightarrow$ \par
			      $\qquad \quad \quad r, y := r \cdot x, y - 1$ \par
			      $\qquad \quad fi$ \par
			      $\qquad od$ \par
			      $\quad \{r =A^B\}$ \par
		      ]\par
	      \end{absolutelynopagebreak}\par

\end{enumerate}

\newpage

\subsubsection{Ejercicio 0}

Derive un programa que calcule $A * B$ donde A y B son números naturales. Solo
se permite el uso de div 2, mod 2, *2, suma y resta.\par

\textbf{Solución}\par

\begin{enumerate}
	\item Se define la función recursiva.\par

	      \begin{equation}
		      F(x,y) =
		      \begin{cases}
			      0               & \text{si } y = 0    \\
			      x + F(x, y - 1) & \text{si } y \neq 0
		      \end{cases}
	      \end{equation} \par

	\item Se define como invariante el siguiente predicado.\par
	      \begin{center}
		      $P: F(A,B) = r + F(x,y) \equiv A \cdot B = r + x \cdot y$\par
	      \end{center}

	\item Valores iniciales.\par
	      Para que el invariante sea cierto antes de la entrada en el bucle, se debe
	      cumplir que $r = 0$ (elemento neutro de la suma, el cual es el operador
	      principal de F), $x = A$, $y = B$. De esta manera, el invariante propuesto
	      queda como se muestra a continuación.\par

	      $A \cdot B = 0 + A \cdot B = A \cdot B$\par
	\item Para validar que el invariante se cumple durante el bucle veamos qué pasa
	      cuando $y \neq 0$.\par
	      $\qquad F(A,B) = r + F(x,y)$\par
	      $\equiv \qquad \langle $ Definición de $ F \rangle$\par
	      $\qquad F(A,B) = r + x + F(x, y - 1)$\par
	      $\equiv \qquad \langle $ S.T sobre P $\rangle$\par
	      $\qquad P(r,y := r + x, y - 1)$\par
	      Las actualizaciones son las siguientes.\par

	      \begin{center}
		      $x,y := r + x, y - 1$\par
	      \end{center}

	      Para validar que el invariante se cumple al finalizar el bucle veamos qué pasa
	      cuando $y = 0$.\par
	      $\qquad F(A,B) = r + F(x,y)$\par
	      $\equiv \qquad \langle $ Definición de $ F \rangle$\par
	      $\qquad F(A,B) = r + 0$\par
	      $\equiv \qquad \langle $ Aritmética $\rangle$\par
	      $\qquad F(A,B) = r$\par

	      Entonces, se dice que $r$ almacena el resultado deseado, al momento de
	      finalizar el bucle.\par

\end{enumerate}

Finalmente, el programa es el siguiente.\par

\begin{absolutelynopagebreak}
	[\par
		$\quad const \; A, B: int;$\par
		$\quad var \; r, x, y : int;$\par
		$\quad r, x, y := 0, A, B;$\par
		$\quad \{Inv:\; A \cdot B = r + x \cdot y \land y \geq 0\}\{Cota: \; y\}$ \par
		$\qquad do \; y \neq 0 \rightarrow$ \par
		$\qquad \quad r, y := r + x, y - 1$ \par
		$\qquad od$ \par
		$\quad \{r =A \cdot B\}$ \par
	]\par
\end{absolutelynopagebreak}\par

\newpage

Finalmente, el programa es el siguiente.\par

\begin{absolutelynopagebreak}
	[\par
		$\quad const \; N: int;$\par
		$\quad var \; n, r : int;$\par
		$\quad n := N;$\par
		$\quad \{Inv:\; fusc(n) = fusc(N) \land N \geq 0\}\{Cota: \; n\}$ \par
		$\qquad do \; n \neq 0 \land n \neq 1 \rightarrow$ \par
		$\qquad \quad if \; n \mod 2 = 0 \rightarrow$ \par
		$\qquad \qquad n := n \; div \; 2$ \par
		$\qquad \quad [\;] \; n \mod 2 = 1 \rightarrow$ \par
		$\qquad \qquad n := n + n \; div \; 2 $ \par
		$\qquad \quad fi$ \par
		$\qquad od$ \par
		$\qquad if \; n = 0 \rightarrow$ \par
		$\qquad \quad r = 0$ \par
		$\qquad [\;] \; n = 1 \rightarrow$ \par
		$\qquad \quad r = 1$ \par
		$\qquad fi \rightarrow$ \par
		$\quad \{r = fusc(N)\}$ \par
	]\par
\end{absolutelynopagebreak}\par

\newpage

\subsubsection{Ejercicio 4}

Resolver.\par

\begin{absolutelynopagebreak}
	[\par
	$\quad const \; N: int;$\par
	$\quad const \; f: array [0..N) of int;$\par
			$\quad var \; r : bool;$\par
			$\quad S$\par
			$\quad \{r \equiv (\exists \; i| 0 \leq i < N: f[i] = 0)\}$ \par
		]\par
\end{absolutelynopagebreak}\par

Definiendo,\par

$G(n) \equiv (\exists i| n \leq i < N: f[i] = 0)$ para $0 \leq n \leq N$\par
y derivando la relación de recurrencia para $G$.\par

\textbf{Solución}\par

\begin{enumerate}
	\item Se define la función recursiva.\par

	      \begin{equation}
		      G(n) \equiv
		      \begin{cases}
			      true     & \text{si } f[n] = 0                   \\
			      false    & \text{si } f[n] \neq 0 \land n = 0    \\
			      G(n - 1) & \text{si } f[n] \neq 0 \land n \neq 0
		      \end{cases}
	      \end{equation} \par

	\item Invariante de cola para este caso.\par
	      \begin{center}
		      $G(n) \equiv G(N) \land 0 \leq n \leq N$
	      \end{center}

	\item El programa final queda.\par
	      \begin{absolutelynopagebreak}
		      [\par
		      $\quad const \; N: int;$\par
		      $\quad const \; f: array [0..N) \; of \; int;$\par
				      $\quad var \; r : bool;$\par
				      $\quad var \; n : int;$\par
				      $\quad \{N \geq 0\}$\par
				      $\qquad n := N;$\par
				      $\qquad \{Inv: G(n) \equiv G(N) \land 0 \leq n \leq N\} \{Cota: n\}$\par
				      $\qquad do \; f[n] \neq 0 \land n \neq 0 \rightarrow$\par
				      $\qquad \quad n := n - 1$\par
				      $\qquad od$\par
				      $\qquad r := f[n] = 0$\par
				      $\quad \{r \equiv (\exists \; i| 0 \leq i < N: f[i] = 0)\}$ \par
			      ]\par
	      \end{absolutelynopagebreak}\par

\end{enumerate}

\newpage

\subsubsection{Ejercicio 2 (página 81)}

La función A está definida en los números naturales.\par

$A(0) = 1$\par
$A(2n) = 2*A(n) \; n \geq 1$\par
$A(2n + 1) = n + A(2n) \; n \geq 0$\par

Derivar un programa para el cálculo de $A(N)$, $N \geq 0$\par

\textbf{Solución}\par

\begin{enumerate}
	\item Se define la función recursiva.\par
	      \begin{equation}
		      A(n) \equiv
		      \begin{cases}
			      1         & \text{si } n = 0                       \\
			      2*A(n)    & \text{si } n \geq 0 \land n \mod 2 = 0 \\
			      n + A(2n) & \text{si } n \geq 1 \land n \mod 2 = 1
		      \end{cases}
	      \end{equation} \par

	\item Invariante de cola.\par
	      \begin{center}
		      $P: F(N) = r + F(n)$
	      \end{center}

	\item Para el cumplimiento de este Invariante antes del bucle las variables r y n
	      deben ser inicializadas como se muestra a continuación.\par
	      \begin{center}
		      $r,n := 0, N$
	      \end{center}
	\item Cuando $n \neq 0$, estamos dentro del bucle. Se tiene dos posibilidades.\par

	      \begin{itemize}
		      \item Si $n \mod 2 = 0$, se debe cumplir el invariante.\par
		            $\qquad F(N) = r + F(n)$\par
		            $\equiv \quad \langle$ Definición de $F \rangle$\par
		            $\qquad F(N) = r + 2*A(n)$\par
	      \end{itemize}

\end{enumerate}

\newpage

\subsection{Ejercicios Tarea 7 Abr-Jul 2015}

\subsubsection{Ejercicio 1}

La definición recursiva de los números de Catalán es

\begin{equation}
	C_n =
	\begin{cases}
		1                              & \text{si } n = 0 \\
		\frac{2(2n - 1)}{n + 1}C_{n-1} & \text{si } n > 0
	\end{cases}
\end{equation} \par

Escriba una función recursiva que implemente en GCL esta definición.\par
Escriba un programa que calcule los primeros N números de catalán y los
alamacene en un arreglo A.\par
Pruebe que la función recursiva es correcta.\par

\textbf{Solución}\par

\begin{enumerate}
	\item La función recursiva se da en el resultado.\par
	      \begin{equation}
		      C(n) =
		      \begin{cases}
			      1                               & \text{si } n = 0 \\
			      \frac{2(2n - 1)}{n + 1}C(n - 1) & \text{si } n > 0
		      \end{cases}
	      \end{equation} \par
	\item El invariante de cola es
	      \begin{center}
		      $P: C(N) = r * C(n)$\par
	      \end{center}
	\item Valores iniciales.\par
	      Dado que el operador principal de $C(n)$ es la múltiplicaicón y para que el
	      invariante sea cierto antes de la entrada en el bucle se debe inicializar con
	      $r = 1$ (elemento neutro de la multiplicación), luego con $n = N$ el invariante
	      se cumple.\par
	      $r * C(n) = 1 * C(N) = C(N)$\par
	\item Se verifica el invariante dentro del bucle cuando $n > 0$.\par
	      $\qquad C(N) = r * C(n)$\par
	      $\equiv \quad \langle$ Definición de  $C(n)\rangle$\par
	      $\qquad C(N) = r * \frac{2(2n - 1)}{n + 1}C(n - 1)$\par
	      $\equiv \quad \langle$ Definición de  $P\rangle$\par
	      $\qquad P(r,n : r * \frac{2(2n - 1)}{n + 1}, n - 1)$\par

	      Las actualizaciones de $r$ y $n$ son las siguientes.\par

	      \begin{center}
		      $r, n := r * \frac{2(2n - 1)}{n + 1}, n - 1$
	      \end{center}

	      Ahora verifiquemos que pasa con el invariante al finalizar el bucle.\par

	      $\qquad C(N) = r * C(n)$\par
	      $\equiv \quad \langle$ S.T $n = 0\rangle$\par
	      $\qquad C(N) = r * C(0)$\par
	      $\equiv \quad \langle$ Definición de $C\rangle$\par
	      $\qquad C(N) = r * 1$\par
	      $\equiv \quad \langle$ Aritmética $\rangle$\par
	      $\qquad C(N) = r$\par

	      Por lo tanto al finalizar el bucle $r$ almacena el valor de $C(N)$.\par

	\item La función que calcula el número de catalán para los primeros N enteros.\par
	      \begin{absolutelynopagebreak}
		      $\textbf{func} \; catalan(N: int) \rightarrow float$ \par
		      $\quad \{pre: N \geq 0\}$ \par
		      $\quad \{post: (\forall k| 0 \leq k < n: )\}$\par
		      [\par
			      $\quad var \; r : float$;\par
			      $\quad var \; n : int$;\par
			      $\qquad r, n := 1, N$;\par
			      $\qquad \{Inv: C(N) = r * C(n)\} \{Cota: n\}$\par
			      $\qquad do \; n \neq 0 \rightarrow$\par
			      $\qquad \quad A[N - n] := r$\par
			      $\qquad \quad r, n := r * \frac{2(2n - 1)}{n + 1}, n - 1$;\par
			      $\qquad od$\par
			      $\quad \{r = C(N)\}$ \par
			      $\qquad >> r$\par
		      ]\par
	      \end{absolutelynopagebreak}\par
	\item Programa que calcula los N primeros números de Catalán y los alamacena en un
	      arreglo A de tamaño N.\par

	      \begin{absolutelynopagebreak}
		      [\par
		      $\quad const \; N : int$;\par
		      $\quad const \; A: array [0..N) \; of \; float;$\par
				      $\quad var \; r : float$;\par
				      $\quad var \; n : int$;\par
				      $\qquad \{N \geq 0\}$;\par
				      $\qquad r, n := 1, N$;\par
				      $\qquad \{Inv: C(N) = r * C(n)\} \{Cota: n\}$\par
				      $\qquad do \; n \neq 0 \rightarrow$\par
				      $\qquad \quad A[N - n] := r$\par
				      $\qquad \quad r, n := r * \frac{2(2n - 1)}{n + 1}, n - 1$;\par
				      $\qquad od$\par
				      $\quad \{r = C(N)\}$ \par
			      ]\par
	      \end{absolutelynopagebreak}\par

\end{enumerate}

\newpage

\subsubsection{Ejercicio 2-c}

Dado dos vectores A y B, halle una función recursiva para decir si son iguales.\par

\textbf{Solución}\par

\begin{absolutelynopagebreak}
	$\textbf{func} \; igualesRec(N: int, n: int, A: array[0..N) \; of \; int, B: array[0..N) \; of \; int) \rightarrow bool$ \par
	$\quad \{pre: 0 \leq n\}$ \par
	$\quad \{post: iguales \equiv (\forall k| 0 \leq k < n: A[k] = B[k])\}$ // Cuando va desde el caso base a los recursivos\par
	[\par
		$\quad var \; r : bool$;\par
		$\quad var \; r := true$;\par
		$\quad if \; n = 0 \rightarrow r := r \land A[n] = B[n]$\par
		$\quad [\;] \; n \neq 0 \rightarrow r := r \land igualesRec(N, n - 1, A, B)$\par
		$\qquad >> r$\par
	]\par
\end{absolutelynopagebreak}\par

Esta función debe ser llamada por otra función de la siguiente forma.\par

\begin{absolutelynopagebreak}
	$\textbf{func} \; iguales(N: int, A: array[0..N) \; of \; int, A: array[0..N) \; of \; int) \rightarrow bool$ \par
	$\quad \{pre: N \geq 0\}$ \par
	$\quad \{post: iguales \equiv (\forall k| 0 \leq k < N: A[k] = B[k])\}$\par
	[\par
		$\quad var \; n : int$;\par
		$\quad var \; res : bool$;\par
		$\quad if \; N = 0 \rightarrow res := true$;\par
		$\quad [\;] \; N > 0 \rightarrow n := N - 1; \;res := igualesRec(N, n, A, B)$;\par
		$\qquad n := N - 1$;\par
		$\qquad res := igualesRec(N, n, A, B)$\par
		$\quad fi$;\par
		$\qquad >> res$\par
	]\par
\end{absolutelynopagebreak}\par

Otra forma de expresar el caso base.\par
\begin{absolutelynopagebreak}
	$\textbf{func} \; igualesRec(N: int, n: int, A: array[0..N) \; of \; int, B: array[0..N) \; of \; int) \rightarrow bool$ \par
	$\quad \{pre: 0 \leq n\}$ \par
	$\quad \{post: iguales \equiv (\forall k| 0 \leq k < n: A[k] = B[k])\}$ // Cuando va desde el caso base a los recursivos\par
	[\par
		$\quad var \; r : bool$;\par
		$\quad if \; n = -1 \rightarrow r := true$\par
		$\quad [\;] \; n \geq 0 \rightarrow r := A[n] = B[n] \land igualesRec(N, n - 1, A, B)$\par
		$\qquad >> r$\par
	]\par
\end{absolutelynopagebreak}\par

\newpage

\subsubsection{Ejercicio 2-e}

Dada una matriz de tamaño NxN devolver la suma de los elementos de la diagonal.\par

\textbf{Solución}\par

\begin{absolutelynopagebreak}
	$\textbf{func} \; sumaDiagRec(N: int, n: int, A: array[0..N)x[0..N) \; of \; int) \rightarrow int$ \par
	$\quad \{pre: n \leq N\}$ \par
	$\quad \{post: sumaDiagRec \equiv (\sum k| 0 \leq k < n: A[k][k])\}$\par
	[\par
		$\quad var \; r: int$;\par
		$\quad var \; r := 0$;\par
		$\quad if \; n = -1 \rightarrow r := 0$\par
		$\quad [\;] \; n \geq 0 \rightarrow r := A[n][n] + sumaDiagRec(N, n - 1, A)$\par
		$\quad fi$\par
		$\qquad >> r$\par
	]\par
\end{absolutelynopagebreak}\par

\newpage

\section{Recursión de Cola}

\subsection{Ejercicios de clase}\par

\subsubsection{Ejercicio de clase 1}

Convertir la función de sumatoria de un arreglo de enteros en una recursión de
cola.\par

\begin{absolutelynopagebreak}
	$\textbf{func} \; sum(N: int, A: array[0..N) \; of \; int, i: int) \rightarrow int$ \par
	[\par
		$\quad var \; r: int$;\par
		$\quad if \; i = N \rightarrow r := 0$\par
		$\quad [\;] \; i < N \rightarrow r := A[i] + sum(A, N, i + 1)$\par
		$\quad fi$\par
		$\qquad >> r$\par
	]\par
\end{absolutelynopagebreak}\par

\textbf{Solución}

\begin{enumerate}
	\item \textbf{Llamadas recursivas}: Identificar la llamada recursiva.\par
	      $sum(A, N, i + 1)$\par
	\item \textbf{Acumuladores}: Definir el acumulador en base a la llamada recursiva, la cual
	      corresponde a la suma de cada uno de los elementos del arreglo desde 0 hasta i
	      esto es posible representarlo matemáticamente como una sumatoria.\par
	      $a = (\sum k | 0 \leq k < i:A[k])$\par
	\item \textbf{Iterador}: Dado que en cada llamada recursiva $i$ aumenta en 1 entonces este $i$
	      puede ser usado como iterador, el cual ya existe en la llamada recursiva.\par
	      Se agrega el acumulador a la definición de la nueva suma.\par

	      $\textbf{func} \; sumCola(N: int, A: array[0..N) \; of \; int, i: int, a: int) \rightarrow int$ \par
	      $\{a = (\sum k | 0 \leq k < i:A[k])\}$ \par

	\item \textbf{Actualización de $a$}: Veamos qué pasa en $i+1$.\par
	      \begin{absolutelynopagebreak}
		      $\qquad (\sum k | 0 \leq k < i:A[k])(i := i + 1)$\par
		      $\equiv \qquad \langle S.T \rangle$\par
		      $\qquad (\sum k | 0 \leq k < i + 1:A[k])$\par
		      $\equiv \qquad \langle$ Último término $\rangle$\par
		      $\qquad (\sum k | 0 \leq k < i:A[k]) + A[i]$\par
		      $\equiv \qquad \langle$ Definición de $a\rangle$\par
		      $\qquad a + A[i]$\par
	      \end{absolutelynopagebreak}

	      Esto corresponde a la expresión de $a$ en $i + 1$.\par
	\item \textbf{Cuerpo de la función}: se transforma la llamada recursiva en una de cola.\par
	      $r := A[i] + sum(A, N, i + 1) \rightarrow r := sumCola(A, N, i + 1, a + A[i])$\par
	\item \textbf{Casos bases}: Se toma el valor final del iterador, cuando $i = N$ se tiene.\par
	      $a = (\sum k | 0 \leq k < N:A[k])$\par
	      Lo cual corresponde a la suma de todos los elementos de A que es lo que
	      soluciona el problema planteado.
	\item \textbf{Valores iniciales}: Se verifica la expresión de a en $i = 0$.\par
	      $a = (\sum k | 0 \leq k < 0:A[k]) = 0$\par
\end{enumerate}

La función finalmente queda.\par

\begin{absolutelynopagebreak}
	$\textbf{func} \; sumCola(N: int, A: array[0..N) \; of \; int, i: int, a: int) \rightarrow int$ \par
	[\par
		$\quad var \; r: int$;\par
		$\quad if \; i = N \rightarrow r := a$\par
		$\quad [\;] \; i < N \rightarrow r := sumCola(A, N, i + 1, a + A[i])$\par
		$\quad fi$\par
		$\qquad >> r$\par
	]\par
\end{absolutelynopagebreak}\par

La suma de todos los elementos de un arreglo $A$ se hace con la siguiente
llamada.\par

$sumCola(N, A, 0, 0)$\par

\newpage

\subsubsection{Ejercicio de clase 2}

Convertir la función de Fibonacci en una recursión de cola.\par

\begin{absolutelynopagebreak}
	$\textbf{func} \; fib(n: int) \rightarrow int$ \par
	[\par
		$\quad var \; r: int$;\par
		$\quad if \; n = 0 \rightarrow r := 0$\par
		$\quad [\;] \; n = 1 \rightarrow r := 1$\par
		$\quad [\;] \; n > 1 \rightarrow r := fib(n - 1) + fib(n - 2)$\par
		$\quad fi$\par
		$\qquad >> r$\par
	]\par
\end{absolutelynopagebreak}\par

\textbf{Solución}

\begin{enumerate}
	\item \textbf{Llamadas recursivas}: Se identifican las llamadas recursivas que en este
	      caso son dos.\par
	      $fib(n - 1), fib(n - 2)$\par
	\item \textbf{Acumuladores}: Como se tienen dos llamadas recursivas se deben definir
	      dos acumuladores.\par
	      $a \rightarrow fib(n - 1)$\par
	      $b \rightarrow fib(n - 2)$\par

	      Se debe definir una expresión para $a$ y $b$ que caracterice estos acumuladores
	      en todo momento del programa.\par
	      Dado que la función ya está definida en $n = 0$ ($fib(0) = 0$) y en $n = 1$
	      ($fib(1) = 1$), entonces es posible hacer el cambio de variable $n := n + 2$
	      quedando las expresiones para $a$ y $b$ de la siguiente manera.\par
	      $a \rightarrow fib(n + 1)$\par
	      $b \rightarrow fib(n)$\par
	      Por último, las expresiones que definen a $a$ y $b$ en todo punto de la
	      recursión son las siguientes.\par
	      $a = fib(N - n + 1) \; \forall \; 0 \leq n \leq N$\par
	      $b = fib(N - n) \; \forall \; 0 \leq n \leq N$\par

	\item \textbf{Iterador}: El iterador es $n$.\par
	      $n: int$\par
	\item Expresión para la nueva función.\par
	      $\textbf{func} \; fibCola(n: int, a: int, b: int) \rightarrow int$ \par
	      $\{a = fib(N - n + 1) \land b = fib(N - n)\}$ \par
	\item \textbf{Valores iniciales}: Dada la expresión matemática de $a$ y $b$ conviene que la
	      función empiece en $n = N$.\par
	      $a = fib(N - N + 1) = fib(1) = 1$\par
	      $b = fib(N - N) = fib(0) = 0$\par
	\item \textbf{Casos bases}: Dado que el valor inicial es $n = N$ entonces se toma $n = 0$ como
	      un caso base, para lo cual se cumple.\par
	      $a = fib(N - 0 + 1) = fib(N + 1)$\par
	      $b = fib(N - 0) = fib(N)$\par
	      De esta manera, notamos que $b$ tendrá el valor final de la función.\par
	      Cuando $n = 1$ se tiene.\par
	      $a = fib(N - 1 + 1) = fib(N)$\par
	      $b = fib(N - 1) = fib(N - 1)$\par
	      En este caso base $a$ contiene el valor deseado.\par
	      Por lo tanto,
	      \begin{center}
		      $n = 0 \Rightarrow b = fib(N)$\par
		      $n = 1 \Rightarrow a = fib(N)$\par
	      \end{center}
	\item \textbf{Actualización de acumuladores}: Como el valor incial de $n$ resultó conveniente
	      en $N$, entonces $n$ es decreciente, por lo tanto, los acumuladores se
	      actualizan en $n = n - 1$.\par
	      Para $a$:\par

	      $\qquad fib(N - n + 1)$\par
	      $\equiv \qquad \langle$ Sustitución Textual $n := n - 1\rangle$\par
	      $\qquad fib(N - (n - 1) + 1)$\par
	      $\equiv \qquad \langle$ Aritmética $\rangle$\par
	      $\qquad fib(N - n + 2)$\par
	      $\equiv \qquad \langle$ Definición fibonacci $fib(n) = fib(n - 1) + fib(n - 2)\rangle$\par
	      $\qquad fib((N - n + 2) - 1) + fib((N - n + 2) - 2)$\par
	      $\equiv \qquad \langle$ Aritmética $\rangle$\par
	      $\qquad fib(N - n + 1) + fib(N - n)$\par
	      $\equiv \qquad \langle$ Definición de $ a, b\rangle$\par
	      $\qquad a + b$\par

	      Para $b$:\par
	      $\qquad fib(N - n)$\par
	      $\equiv \qquad \langle$ Sustitución Textual $n := n - 1\rangle$\par
	      $\qquad fib(N - (n - 1))$\par
	      $\equiv \qquad \langle$ Aritmética $\rangle$\par
	      $\qquad fib(N - n + 1)$\par
	      $\equiv \qquad \langle$ Definición de $a\rangle$\par
	      $\qquad a$\par

	      Por lo tanto, las actualizaciones de los acumuladores son las siguientes.\par
	      \begin{center}
		      $a, b := a + b, a$\par
	      \end{center}
\end{enumerate}

Finalmente, la función de fibonacci en recursión de cola queda como sigue.\par

\begin{absolutelynopagebreak}
	$\textbf{func} \; fibCola(n: int, a: int, b: int) \rightarrow int$ \par
	[\par
		$\quad var \; r: int$;\par
		$\quad if \; n = 0 \rightarrow r := b$\par
		$\quad [\;] \; n = 1 \rightarrow r := a$\par
		$\quad [\;] \; n > 1 \rightarrow r := fibCola(n - 1, a + b, a)$\par
		$\quad fi$\par
		$\qquad >> r$\par
	]\par
\end{absolutelynopagebreak}\par

El cáculo de fibonacci para cualquier entero $N$ se obtiene al llamar a la
función como sigue, usando los valores iniciales calculados.\par

\begin{center}
	$fibCola(N, 1, 0)$\par
\end{center}

\newpage

\subsubsection{Ejercicio de practica Ene-Mar 2025}

Usando la técnica de recursión de cola se quiere que derive uan fución
recursiva de cola a partir de la siguiente función recursiva.\par

\begin{absolutelynopagebreak}
	$\textbf{func} \; sumaDigits(n: int) \rightarrow int$ \par
	[\par
		$\quad var \; r: int$;\par
		$\quad var \; r := 0$;\par
		$\quad if \; n = 0 \rightarrow r := 0$\par
		$\quad [\;] \; n > 0 \rightarrow r := n \mod 10 + sumaDigits(n \; div \; 10)$\par
		$\quad fi$\par
		$\qquad >> r$\par
	]\par
\end{absolutelynopagebreak}\par

\textbf{Solución}

\begin{enumerate}
	\item Identificar las llamadas recursivas.\par
	      $sumaDigits(n \; div \; 10)$\par
	\item Se define el parámetro $a$ que representará la llamada recursiva.\par
	      $a = (\sum k | 0)$\par
\end{enumerate}

\newpage

\subsection{Ejercicios Kaldewaij}\par

\subsubsection{Ejercicio 3}

La función fusc está definida en los números naturales.\par

$fusc(0) = 0, fusc(1) = 1$\par
$fusc(2*n) = fusc(n)$\par
$fusc(2*n + 1) = fusc(n) + fusc(n + 1)$\par

Derive un programa que calcule fusc(N) con $N \geq 0$.

\textbf{Solución}\par

\begin{enumerate}
	\item Se define la función recursiva.\par

	      \begin{equation}
		      fusc(n) =
		      \begin{cases}
			      0                     & \text{si } n = 0                                      \\
			      1                     & \text{si } n = 1                                      \\
			      fusc(n)               & \text{si } n \neq 0 \land n \neq 1 \land n \mod 2 = 0 \\
			      fusc(n) + fusc(n + 1) & \text{si } n \neq 0 \land n \neq 1 \land n \mod 2 = 1
		      \end{cases}
	      \end{equation} \par

	\item Se define como invariante el siguiente predicado.\par
	      \begin{center}
		      $P: fusc(n) = fusc(N) $\par
		      $Q: N \geq 0 $\par
	      \end{center}
\end{enumerate}

\newpage

\end{document}
