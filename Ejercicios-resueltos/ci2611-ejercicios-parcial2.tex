\documentclass[hidelinks]{article}
%Packages
%\usepackage{fontawesome}
\usepackage{amsmath}
\usepackage{amssymb}
%\usepackage{mathabx}
\usepackage{dirtytalk}
\usepackage{textcomp}
\usepackage[margin=1in]{geometry}
\usepackage{tikz}
\usetikzlibrary{arrows,positioning,shapes,fit,calc}
\pgfdeclarelayer{background}
\pgfsetlayers{background,main}
\usepackage{hyperref}
\usepackage{setspace}
\usepackage{framed}
\usepackage{fancyhdr}
\pagestyle{fancy}
\setlength{\headheight}{13.07225pt}
\rfoot[CI2611 - Problemario]{CI2611 - Problemario - p2}
%\rfoot{\thepage}
\doublespacing
\usepackage{needspace}
\usepackage{array}
\usepackage{booktabs}
\usepackage{indentfirst}
%---
%Customize
\newcommand{\myparagraph}[1]{\paragraph{#1}\mbox{}\\}
\newcommand\myrule[1]{\multicolumn{1}{| l}{#1}}
\def\multiset#1#2{\ensuremath{\left(\kern-.3em\left(\genfrac{}{}{0pt}{}{#1}{#2}\right)\kern-.3em\right)}}
\renewcommand{\contentsname}{Contenido}
%---

%--- No page break
\newenvironment{absolutelynopagebreak}
{\Needspace{10\baselineskip}\begin{quote}}
		{\end{quote}}
%---

\title{CI2611 - Algoritmos y estructuras I \\ Parcial 2 }%\\ \href{https://github.com/your-username/your-repo}{\faGithub}}
\date{Abr-Jul 2025}
\author{Daniel Delgado}
\begin{document}
\maketitle

\tableofcontents

\newpage

\section{Resumen Parcial 2}

\subsection{Técnicas de derivación de Invariantes}\par

\begin{enumerate}
	\item Eliminar un predicado de una conjunción.\par
	      A partir de una postcondición de la forma: $P \land Q$\par
	      Es posible elegir:\par
	      Inv: $P$\par
	      Guarda: $\neg Q$\par
	      El programa resultante.\par
	      \begin{absolutelynopagebreak}
		      $\{Inv: P\}$\par
		      $\quad do \; \neg Q \rightarrow$\par
		      $\qquad S$\par
		      $\quad od$\par
		      $\{P \land Q\}$\par
	      \end{absolutelynopagebreak}
	\item Reemplazo de constantes por variables.\par
	      A partir de una postcondición de la forma: $r = F(N)$\par
	      Es posible elegir:\par
	      Inv: $r = F(x) \land 0 \leq x \leq N$\par
	      El programa resultante.\par
	      \begin{absolutelynopagebreak}
		      $\{Inv: r = F(x) \land 0 \leq x \leq N\}$\par
		      $\qquad S$\par
		      $\{r = F(N)\}$\par
	      \end{absolutelynopagebreak}
	\item Fortalecimiento de invariantes.\par
	      A partir de una postcondición de la forma: $r = F(N)$\par
	      Es posible elegir:\par
	      Inv: $r = F(x)$\par
	      Cuando la función $F(x)$ es difícil de calcular en una sola iteración se debe
	      determinar una función $G(x)$ tal que cumpla.\par
	      $r = F(x) \otimes G(x)$, donde $\otimes$ representa la operación que une a las expresiones $F$ y $G$.\par
	      Y podemos fortalecer el invariante con un nuevo predicado.\par
	      $s = G(x)$\par
	      De manera que r se vuelve recursiva.\par
	      $r = r \otimes s$\par
	      Finalmente el invariante queda.\par
	      $r = F(x) \land 0 \leq x < N \land s = G(x)$\par
	      En este punto si $G(x)$ sigue siendo difícil de computar en una sola iteración,
	      entonces se debe aplicar el mismo procedimiento de forma recurrente hasta
	      alcanzar una expresión fácil de calcular en una iteración.\par
\end{enumerate}

\subsection{Arreglos}\par

Notación declaración de arreglos:\par
$a: array \; [p..q) \; of \; t$, donde t es el tipo de dato de cada elemento\par
$a: array \; [p..q) \; of \; array \; [r..s) \; of \; t$, Matríz\par
$a: array \; [p..q) \times [r..s) \; of \; t$, Matríz\par

\textbf{Regla de asignación.}\par

$\{P\} \; a[E] := F \; \{Q\}$, con $a: array [p..q) \; of \; t$ se traduce matemáticamente en:\par

\begin{equation}
	a[i: E][j] =
	\begin{cases}
		a[j] & \text{si } i \neq j \\
		E    & \text{si } i = j
	\end{cases}
\end{equation} \par

\textbf{Regla de correctitud para la asignación.}\par
$\{P\} \; a[E] := F \; \{Q\} \iff [P \Rightarrow p \leq E < q \land Q(a := a(E: F))]$\par

\textbf{Regla de correctitud para el intercambio de elementos en un arreglo.}\par
$\{P\} \; Intercambio(a,E,F) \; \{Q\} \iff [P \Rightarrow p \leq E < q \land p \leq F < q \land Q(a := a(E, F: a[F], a[E]))]$\par

\subsection{Procedimientos}\par
Notación:\par
\begin{absolutelynopagebreak}
	$\textbf{proc} \; <nombre\_procedimiento>(<parametro1>; <parametro2>; \dots; <parametroN>)$ \par
	$\quad \{precondicion: P\}$ \par
	$\quad \{postcondicion: Q\}$\par
	$\qquad S$\par
\end{absolutelynopagebreak}

Donde, $<parametro> = <tipo \; parametro> \; <nombre\_variable: tipo>$\par
y $<tipo \; parametro>$ puede ser: in (entrada), in-out (entrada-salida), out
(salida).\par

\vspace{2em}

\textbf{Definición procedimiento general.}\par

\begin{absolutelynopagebreak}
	$\textbf{proc} \; p \; (in \; x; \; in-out \; y; \; out \; z)$ \par
	$\quad \{P_{def}\}$ \par
	$\quad \{Q_{def}\}$\par
	$\qquad S$\par
\end{absolutelynopagebreak}

\textbf{Correctitud de la llamada a procedimientos.}\par
\textit{Nota: En este resumen solo se plasmará el caso general que en la literatura es}
conocido como el caso B2.\par
Con el procedimiento p definido anteriormente se desea probar la siguiente
tripleta.\par
\begin{center}
	$\{P_{llam}\} \; p(E, a, b) \; \{Q_{llam}\}$
\end{center}\par
Para lo cual se deben cumplir los siguientes predicados.\par
$1.[P_{llam} \Rightarrow P_{def}(x,y := E, a)]$\par
$2.[P_{llam}(a,b := A,B) \land Q_{def}(x,y_0,y,z := E(a,b := A,B), A, a, b) \Rightarrow Q_{llam}]$\par

Donde $A,B$ son los valores iniciales de los argumentos de entrada-salida y
salida respectivamente y son constantes cualquieras dentro de la demostración.\par

Dado que un procedimiento puede tener múltiples parámetros de entrada,
entrada-salida y salida, tanto los parámetros $x,y,z$ como los argumentos
$E,a,b$ represnetan una lista y no una única variable, bajo la misma
justificación las constantes $A,B$ represnetan una lista de constantes de los
estados de los argumentos de entradas-salida y salida, respectivamente.\par

\subsection{Funciones}\par
Notación:\par
\begin{absolutelynopagebreak}
	$\textbf{func} \; f(<param1>; <param2>; \dots; <paramN>) \rightarrow tipoRetorno$ \par
	$\quad \{precondicion: P\}$ \par
	$\quad \{postcondicion: Q\}$\par
	$\quad [$\par
			$\qquad I;$\par
			$\qquad >> E$\par
			$\quad ]$\par
\end{absolutelynopagebreak}

Donde $I$ representa el conjunto de instrucciones que ejecuta la función y $E$
representa el valor retornado.\par

\textbf{Correctitud de funciones.}\par
Para probar una función es necesario traducir la función a un procedimiento.\par

\begin{absolutelynopagebreak}
	$\textbf{proc} \; pf(<param1>; <param2>; \dots; <paramN>; out \; \theta: tipoRetorno)$ \par
	$\quad \{precondicion: P\}$ \par
	$\quad \{postcondicion: Q\}$\par
	$\quad [$\par
			$\qquad I;$\par
			$\qquad \theta := E$\par
			$\quad ]$\par
\end{absolutelynopagebreak}

Donde, $\theta$ es el parámetro que representa a una variable en el cuerpo del
programa principal que pasará como argumento durante la llamada del nuevo
procedimiento y recibirá el valor de retorno de la función definida
originalmente.\par

\textbf{Algunos ejemplos.}\par

\begingroup
\setlength\tabcolsep{3pt}
\small
\begin{tabular}{b{2in}c  b{2in}r}
	\toprule
	{\bfseries Función}                                                      &  & {\bfseries Procedimiento}                                                                                         \\
	\midrule
	$x := 3*(f(a) + b)$                                                      &  & $pf(a,\phi); \; x := 3*(\phi + b)$                                                                                \\
	\midrule
	$if \; f(a) \geq 0 \rightarrow b$\newline $[\;] f(b) \leq 0 \rightarrow c$ &  & $pf(a, \phi);pf(b,\omega)$ \newline $if \; \phi \geq 0 \rightarrow b$\newline $[\;] \omega \leq 0 \rightarrow c$    \\
	\midrule
	$do \; n - f(i) > 0 \rightarrow \dots $\newline $\dots$ \newline $od$    &  & $pf(i, \theta);$ \newline $do \; n - \theta > 0 \rightarrow \dots $\newline $\dots; pf(i,\theta)$ \newline $od$ & \\
	\midrule
\end{tabular}
\endgroup

\newpage
\vspace*{\fill}
	\hspace*{\fill} --- PROBLEMAS RESUELTOS --- \hspace*{\fill}
\vspace*{\fill}
\thispagestyle{empty}
\newpage

\section{Derivación de Invariantes}

\subsection{Ejercicios sección 4.3 Kaldewaij} \par

Derivar programa.\par

% ************* Start Exercise ******************

\subsubsection{Ejercicio 0}
[\par
$\quad const N: int$\par
$\quad const A: array \; [0..N) \; of \; int$\par
$\quad var \; r: int$\par
$\quad \{N \geq 1\}$\par
\hspace{1em} S \par
$\quad \{r =(\max \; p,q| \; 0 \leq p < q < N : A[p] - A[q])\}$ \par
]

\newpage

\textbf{Solución}

\begin{enumerate}
	\item Se define el siguiente invariante: \par
	      \begin{center}
		      P0: $r =(\max \; p,q| \; 0 \leq p < q < x : A[p] - A[q])$ \par
		      P1: $0 \leq x \leq N$ \par
		      $x=0 \Rightarrow (\max \; p,q| \; 0 \leq p < q < 0 : A[p] - A[q])=- \infty$ \par
	      \end{center}
	\item Guarda $x<N$
	\item Se verifica la actualización de r en $x+1$ \par
	      $ \qquad (\max \; p,q| \; 0 \leq p < q < x+1 : A[p] - A[q])$ \par
	      $\equiv  \qquad \langle Ultimo \; termino: \; q = x \rangle$ \par
	      $ \qquad (\max \; p,q| \; 0 \leq p < q < x : A[p] - A[q]) \; \max \; (\max \; p| \; 0 \leq p < x : A[p] - A[x])$ \par
	      $\equiv  \qquad \langle P0 \rangle$ \par
	      $ \qquad r \; \max \; (\max \; p| \; 0 \leq p < x : A[p] - A[x])$ \par
	      $\equiv  \qquad \langle  \;$ Distributividad - sobre max $\rangle$ \par
	      $ \qquad r \; \max \; ((\max \; p| \; 0 \leq p < x : A[p]) - A[x])$ \par

	\item Introducimos una nueva variable \par
	      $ \qquad s = (\max \; p| \; 0 \leq p < x : A[p])$ \par
	      De esta manera se fortalece el invariante con el predicado.\par
	      \begin{center}
		      $ Q: s = (\max \; p| \; 0 \leq p < x : A[p])$ \par
		      $x = 0 \Rightarrow s = (\max \; p| \; 0 \leq p < 0 : A[p])=- \infty$ \par
	      \end{center}
	      La actualización de r queda como sigue.\par
	      \begin{center}
		      $r = r \; \max \; (s - A[x])$
	      \end{center}
	\item Ahora evaluamos la actualizacion de s en $x+1$ \par
	      $ \qquad (\max \; p| \; 0 \leq p < x+1 : A[p])$ \par
	      $\equiv  \qquad \langle Ultimo \; termino \rangle$ \par
	      $ \qquad (\max \; p| \; 0 \leq p < x : A[p]) \; \max \; A[x]$ \par
	      $\equiv  \qquad \langle Q \rangle$ \par
	      $ \qquad s \; \max \; A[x]$ \par
	      La actualización de s queda: \par
	      \begin{center}
		      $ s = s \; \max \; A[x]$ \par
	      \end{center}

\end{enumerate}

Programa: \par
[\par
$\quad const \; N: int;$\par
$\quad const \; A: array \; [0..N) \; of \; int;$\par
$\quad var \; r,s,x: int;$\par
$\quad \{N \geq 1\}$\par
$\quad x,r,s := 0,-\infty,-\infty $ \par
$\quad \{Inv: r =(\max \; p,q| \; 0 \leq p < q < x : A[p] - A[q]) \land 0 \leq x \leq N \land s = (\max \; p| \; 0 \leq p < x : A[p])\} \newline \{Cota: N-x\}$\par
\hspace{1em} do $x < N \rightarrow$ \par
$\qquad r = r \; \max \; (s - A[x]);$ \par
$\qquad s = s \; \max \; A[x];$ \par
$\qquad x = x+1$ \par
\hspace{1em} od \par
$\quad \{r =(\max \; p,q| \; 0 \leq p < q < N : A[p] - A[q])\}$ \par
]\par

\newpage

% ************* Start Exercise ******************

\subsubsection{Ejercicio 4}
[\par
$\quad const N: int;$\par
$\quad const A: array \; [0..N) \; of \; bool;$\par
$\quad var \; r: bool$\par
$\quad \{N \geq 0\}$\par
\hspace{1em} S \par
$\quad \{r \equiv (\exists \; p| \; 0 \leq p \leq N : (\forall i \;|\; 0 \leq i < p:A[i]) \land (\forall i \;|\; p \leq i < N: \neg A[i]))\}$ \par
]

\newpage

\textbf{Solución}

\begin{enumerate}
	\item Se define el siguiente invariante: \par
	      $P0: r \equiv (\exists \; p| \; 0 \leq p \leq x : (\forall i \;|\; 0 \leq i < p:A[i]) \land (\forall i \;|\; p \leq i < x: \neg A[i]))$ \par
	      $P1: 0 \leq x \leq N$ \par
	      $x = 0 \Rightarrow r \equiv $ \par
	      $\qquad (\exists \; p| \; 0 \leq p \leq 0 : (\forall i \;|\; 0 \leq i < p:A[i]) \land (\forall i \;|\; p \leq i < 0: \neg A[i]))$ \par
	      $\equiv  \qquad \langle 0 \leq p \leq 0 \equiv p = 0 \rangle$ \par
	      $\qquad (\exists \; p| \; p=0 : (\forall i \;|\; 0 \leq i < p:A[i]) \land (\forall i \;|\; p \leq i < 0: \neg A[i]))$ \par
	      $\equiv  \qquad \langle Logica \rangle$ \par
	      $\qquad (\forall i \;|\; 0 \leq i < 0:A[i]) \land (\forall i \;|\; 0 \leq i < 0: \neg A[i])$ \par
	      $\equiv  \qquad \langle $ Rango vacío$ \rangle$ \par
	      $\qquad true$ \par
	\item Guarda $x \neq N$
	\item Se verifica la actualización de r en x+1. \par
	      $\qquad (\exists \; p| \; 0 \leq p \leq x+1 : (\forall i \;|\; 0 \leq i < p:A[i]) \land (\forall i \;|\; p \leq i < x+1: \neg A[i]))$ \par
	      $\equiv  \qquad \langle$ Ultimo termino en  $\forall \; i=x \rangle$ \par
	      $\qquad (\exists \; p| \; 0 \leq p \leq x+1 : (\forall i \;|\; 0 \leq i < p:A[i]) \land (\forall i \;|\; p \leq i < x: \neg A[i]) \land \neg A[x])$ \par
	      $\equiv  \qquad \langle$ Dado  $\neg ocurrelibre('p',\neg A[x]) \equiv true \rangle$ \par
	      $\qquad (\exists \; p| \; 0 \leq p \leq x+1 : (\forall i \;|\; 0 \leq i < p:A[i]) \land (\forall i \;|\; p \leq i < x: \neg A[i])) \land \neg A[x]$ \par
	      $\equiv  \qquad \langle$ Ultimo termino en  $\exists \; p=x+1 \rangle$ \par
	      $\qquad ((\exists \; p| \; 0 \leq p \leq x : (\forall i \;|\; 0 \leq i < p:A[i]) \land (\forall i \;|\; p \leq i < x: \neg A[i])) \lor (\forall i \;|\; 0 \leq i < x+1:A[i]) \land (\forall i \;|\; x+1 \leq i < x: \neg A[i])) \land \neg A[x]$ \par
	      $\equiv  \qquad \langle x+1 \leq i < x \equiv false \rangle$ \par
	      $\qquad ((\exists \; p| \; 0 \leq p \leq x : (\forall i \;|\; 0 \leq i < p:A[i]) \land (\forall i \;|\; p \leq i < x: \neg A[i])) \lor (\forall i \;|\; 0 \leq i < x+1:A[i]) \land (\forall i \;|\; false: \neg A[i])) \land \neg A[x]$ \par
	      $\equiv  \qquad \langle$ Rango vacío  $ \rangle$ \par
	      $\qquad ((\exists \; p| \; 0 \leq p \leq x : (\forall i \;|\; 0 \leq i < p:A[i]) \land (\forall i \;|\; p \leq i < x: \neg A[i])) \lor (\forall i \;|\; 0 \leq i < x+1:A[i]) \land true) \land \neg A[x]$ \par
	      $\equiv  \qquad \langle P0 \rangle$ \par
	      $\qquad (r \lor (\forall i \;|\; 0 \leq i < x+1:A[i])) \land \neg A[x]$ \par
	      $\equiv  \qquad \langle$ Ultimo termino en  $\forall \; i=x \rangle$ \par
	      $\qquad (r \lor ((\forall i \;|\; 0 \leq i < x:A[i]) \land A[x])) \land \neg A[x]$ \par

	\item Introducimos una nueva variable. \par
	      $s \equiv (\forall i \;|\; 0 \leq i < x:A[i])$ \par
	      $x=0 \Rightarrow s \equiv (\forall i \;|\; 0 \leq i < 0:A[i])=true$ \par

	      Fortalecemos el invariante con el predicado.

	      \begin{center}
		      $Q: s \equiv (\forall i \;|\; 0 \leq i < x:A[i])$ \par
	      \end{center}

	      De esta manera, la actualización de r queda como sigue. \par

	      \begin{center}
		      $r \equiv (r \lor (s \land A[x])) \land \neg A[x]$ \par
	      \end{center}

	\item Verificacmos la actualización de s en x+1 \par
	      $\qquad (\forall i \;|\; 0 \leq i < x+1:A[i])$ \par
	      $\equiv  \qquad \langle$ Ultimo termino   $\ i=x \rangle$ \par
	      $\qquad (\forall i \;|\; 0 \leq i < x:A[i]) \land A[x]$ \par
	      $\equiv  \qquad \langle Q \rangle$ \par
	      $\qquad s \land A[x]$ \par

	      La actualización de s queda como sigue.

	      \begin{center}
		      $\qquad s \equiv s \land A[x]$ \par
	      \end{center}

\end{enumerate}

\begin{absolutelynopagebreak}
	Programa: \par
	[\par
	$\quad const \; N: int;$\par
	$\quad const \; A: array \; [0..N) \; of \; bool;$\par
	$\quad var \; r,s,x: bool$\par
	$\quad \{N \geq 0\}$\par
	$\quad x,r,s := 0,true,true $ \par
	$\quad \{Inv: r \equiv (\exists \; p| \; 0 \leq p \leq x : (\forall i \;|\; 0 \leq i < p:A[i]) \land (\forall i \;|\; p \leq i < x: \neg A[i])) \land s \equiv (\forall i \;|\; 0 \leq i < x:A[i]) \land 0 \leq x \leq N \} \{Cota: N-x\}$\par
	\hspace{1em} do $x \neq N \rightarrow$ \par
	$\qquad r \equiv (r \lor (s \land A[x])) \land \neg A[x];$ \par
	$\qquad s \equiv s \land A[x];$ \par
	$\qquad x = x+1$ \par
	\hspace{1em} od \par
	$\quad \{r \equiv (\exists \; p| \; 0 \leq p \leq N : (\forall i \;|\; 0 \leq i < p:A[i]) \land (\forall i \;|\; p \leq i < N: \neg A[i]))\}$ \par
	]\par
\end{absolutelynopagebreak}

\newpage

% ************* Start Exercise ******************

\subsection{Parcial 2 Ene-Mar 2025}

\subsubsection{Ejercicio 4}
Sea las funciones F y G. \par

\begin{equation}
	F(n)=
	\begin{cases}
		-10             & \text{si } n=0   \\
		5-F(n-1)+G(n-1) & \text{si } n > 0
	\end{cases}
\end{equation} \par

\begin{equation}
	G(n)=
	\begin{cases}
		0          & \text{si } n=0   \\
		G(n-1) + 7 & \text{si } n > 0
	\end{cases}
\end{equation} \par

[\par
	$\quad const \; N: int;$\par
	$\quad var \; r: int$\par
	$\quad\{N \geq 0\}$\par
	\hspace{1em}     S \par
	$\quad \{r = 2  F(N)\}$ \par
]

\newpage

\textbf{Solución}

\begin{enumerate}
	\item Se define el siguiente invariante. \par
	      \begin{center}
		      $P0: r = 2 F(n)$ \par
		      $P1: 0\leq n \leq N$ \par
		      $n = 0 \Rightarrow r = 2F(0)=-10$ \par
	      \end{center}
	\item Guarda: $x<N$ \par
	\item Actualización de r en $n+1$ \par
	      $\qquad 2 F(n+1)$ \par
	      = $\qquad \langle$ Definición de F $\rangle$\par
	      $\qquad 2(5-F(n+1-1)+G(n+1-1))$ \par
	      = $\qquad \langle$ Aritmética $\rangle$\par
	      $\qquad 10-2F(n)+2G(n)$ \par

	\item Se introduce una nueva variable. \par
	      $s = G(n)$ \par
	      $n=0 \Rightarrow s = G(0)=0$ \par
	      La actualización de r queda. \par

	      \begin{center}
		      $r = 10 - 2r + 2s$ \par
	      \end{center}

	      Se fortalece el invariante con el predicado.
	      \begin{center}
		      $Q: s = G(n)$ \par
	      \end{center}

	\item Se verifica la actualización de s en $n+1$. \par
	      $\qquad G(n+1)$ \par
	      = $\qquad \langle$ Definición de G $\rangle$\par
	      $\qquad G(n+1-1)+7$ \par
	      = $\qquad \langle$ Aritmética $\rangle$\par
	      $\qquad G(n)+7$ \par
	      = $\qquad \langle Q \rangle$\par
	      $\qquad s+7$ \par

	      Actualización de s.
	      \begin{center}
		      $\qquad s = s+7$ \par
	      \end{center}

	      Programa: \par
	      [\par
		      $\quad const \; N: int;$\par
		      $\quad var \; r,s,n: int$\par
		      $\quad\{N \geq 0\}$\par
		      $\quad r,s,n := -10,0,0;$\par
		      $\quad\{inv: r = 2 F(n) \land s = G(n) \land 0 \leq n \leq N\}\{cota: N-n\}$\par
		      $\quad do \; x < N \rightarrow $ \par
		      $\qquad r = 10 - 2r + 2s; $ \par
		      $\qquad s = s + 7; $ \par
		      $\qquad n = n+1 $ \par
		      $\quad od $ \par
		      $\quad \{r = 2  F(N)\}$ \par
	      ]

\end{enumerate}

\newpage

\subsection{Tarea 6 2015}

% ************* Start Exercise ******************

Derive programa. \par

\subsubsection{Ejercicio 1-b}

[\par
	$\quad const N: int;$\par
	$\quad var \; s: int$\par
	$\quad\{N > 0\}$\par
	$\quad Programa $ \par
	$\quad \{s = (\sum i| 0\leq i < N :2^i)\}$ \par
]

\textbf{Solución}

\begin{enumerate}
	\item Se propone las siguientes proposiciones como invariantes.\par
	      \begin{center}
		      $P0: s = (\sum i| 0\leq i < x :2^i)$ \par
		      $P1: 0 \leq x \leq N$ \par
	      \end{center}
	      Si $x = 0 \Rightarrow s = (\sum i| 0\leq i < 0 :2^i) = 0$ \par
	\item Guarda $x < N$.
	\item Se verifica la actualización de s en $x+1$. \par
	      $\quad (\sum i| 0\leq i < x+1 :2^i)$ \par
	      $\equiv \quad \langle$ Último término $ i = x \rangle$ \par
	      $\quad (\sum i| 0\leq i < x :2^i) + 2^x$ \par
	      $\equiv \quad \langle P0 \rangle$ \par
	      $\quad s + 2^x$ \par
	\item Dado que el calculo de $2^x$ no es posible hacerlo en una solo iteración se
	      propone la siguiente expresión. \par
	      $ r = 2^x$ \par
	      Si $x = 0 \Rightarrow r = 1$ \par
	      De esta manera se puede fortalecer el invariante con el predicado a
	      continuación.\par
	      \begin{center}
		      $ Q: r = 2^x$ \par
	      \end{center}
	      Luego, la actualización de s queda como sigue.\par
	      \begin{center}
		      $s = s + r$ \par
	      \end{center}
	\item Verificamos la actualización de r en x+1. \par
	      $\quad 2^{x+1}$ \par
	      $\equiv \quad \langle$ Definición de exponente de la suma $ \rangle$ \par
	      $\quad 2^{x} \cdot 2$ \par
	      $\equiv \quad \langle Q \rangle$ \par
	      $\quad r \cdot 2$ \par

	      De esta manera, la actualización de r es la siguiente. \par

	      \begin{center}
		      $r = 2r$ \par
	      \end{center}

\end{enumerate}

\begin{absolutelynopagebreak}
	Programa final.\par

	[\par
		$\quad const N: int;$\par
		$\quad var \; s,r,x: int$\par
		$\quad\{N > 0\}$\par
		$\quad x,s,r := 0,0,1;$\par
		$\quad\{inv: s = (\sum i| 0\leq i < x :2^i) \land r = 2^x \land 0 \leq x \leq N\}\{cota: N-x\}$\par
		$\quad do \; x < N \rightarrow $ \par
		$\qquad s,r = s + r,2r $ \par
		$\quad od $ \par
		$\quad \{s = (\sum i| 0\leq i < N :2^i)\}$ \par
	]
\end{absolutelynopagebreak}

\newpage

% ************* Start Exercise ******************

\subsubsection{Ejercicio 1-c}

[\par
	$\quad const \; N: int;$\par
	$\quad const \; X: float;$\par
	$\quad var \; s: float$\par
	$\quad\{N > 0\}$\par
	$\quad Sumatoria $ \par
	$\quad \{s = (\sum i| 0 < i < N : \frac{X^i}{i!})\}$ \textcolor{red}{Cambié el enunciado para no incluir el 0} \par
]

\textbf{Solución}

\begin{enumerate}
	\item Se propone el siguiente invariante. \par
	      \begin{center}
		      $P0: s = (\sum i| 0 < i < n : \frac{X^i}{i!})$ \par
		      $P1: 0 < n \leq N$ \par
	      \end{center}
	      $n=1 \Rightarrow s = (\sum i| 0 < i < 1: \frac{X^i}{i!})=0$ \par
	\item Guarda $n < N$. \par
	\item Dado que esta expresión del invariante no es fácil de calcular en una iteración
	      se verifica la actualización de s en $n+1$ para verificar si podemos hallar una
	      función G(n) que sea más fácil de calcular.\par
	      $\quad (\sum i| 0\leq i < n+1 : \frac{X^i}{i!})$ \par
	      $\equiv \quad \langle$ Último término $ i = n \rangle$ \par
	      $\quad (\sum i| 0\leq i < n : \frac{X^i}{i!})+\frac{X^n}{n!}$ \par
	      $\equiv \quad \langle P0 \rangle$ \par
	      $\quad s+\frac{X^n}{n!}$ \par
	\item La expresión anterior es una expresión que no resulta fácil de calcular en una
	      iteración dado que tiene un factorial y una potencia del iterador. Se introduce
	      la siguiente variable.\par
	      $r = \frac{X^n}{n!}$ \par
	      $n = 1 \Rightarrow r = \frac{X^1}{1!}=X$ \par
	      Se fortalece el invariante con la siguiente proposición.

	      \begin{center}
		      $Q: r = \frac{X^n}{n!}$ \par
	      \end{center}

	      Con lo cual la actualización de s queda como sigue. \par
	      \begin{center}
		      $s = s+r$ \par
	      \end{center}

	\item Evaluamos la actualización de r en $n+1$. \par
	      $\quad \frac{X^{n+1}}{(n+1)!}$ \par
	      $\equiv \quad \langle$ Aritmética $ \rangle$ \par
	      $\quad \frac{X \cdot X^n}{(n+1) \cdot n!}$ \par
	      $\equiv \quad \langle$ Aritmética $ \rangle$ \par
	      $\quad \frac{X}{(n+1)} \cdot \frac{X^n}{n!}$ \par
	      $\equiv \quad \langle Q \rangle$ \par
	      $\quad \frac{X}{(n+1)} \cdot r$ \par

	      Se tiene que la expresión anterior resulta fácil de calcular en una iteración
	      por lo tanto no tenemos que continuar fortaleciendo el invariante.\par
	      La actualización de r queda como se muestra.\par

	      \begin{center}
		      $r = \frac{X}{(n+1)} \cdot r$ \par
	      \end{center}

\end{enumerate}

\newpage

\begin{absolutelynopagebreak}
	Programa derivado.\par

	[\par
		$\quad const \; N: int;$\par
		$\quad const \; X: float;$\par
		$\quad var \; s,r: float$\par
		$\quad var \; n: int$\par
		$\quad\{N > 0\}$\par
		$\quad\{inv: s = (\sum i| 0\leq i < n : \frac{X^i}{i!}) \land r = \frac{X^n}{n!} \cdot r \land 0 < n \leq N \}\{cota: N-n\}$\par
		$\quad do \; n < N \rightarrow $ \par
		$\qquad s,r = s + r,\frac{X}{(n+1)} \cdot r $ \par
		$\quad od $ \par
		$\quad \{s = (\sum i| 0 < i < N : \frac{X^i}{i!})\}$ \par
	]
\end{absolutelynopagebreak}



\newpage

% ************* Start Exercise ******************

\subsubsection{Ejercicio 1-f}

[\par
$\quad const $\par
$\qquad N: int;$\par
$\qquad S: array[0..N) \; of \; int;$\par
$\quad var \; r: int$\par
$\quad\{N \geq 0\}$\par
$\quad Programa $ \par
$\quad \{r = (\# i,j| 0\leq i < j < N :S[i] \leq 0 \land S[j] \geq 0)\}$ \par
]

\textbf{Solución}

\begin{enumerate}
	\item Se propone el siguiente invariante. \par
	      \begin{center}
		      $P0: r = (\# i,j| 0\leq i < j < x :S[i] \leq 0 \land S[j] \geq 0)$ \par
		      $P1: 0 \leq x \leq N$ \par
	      \end{center}
	      $x = 0 \Rightarrow r = (\# i,j| 0\leq i < j < 0 :S[i] \leq 0 \land S[j] \geq 0) = 0$ \par

	\item Guarda $x<N$ y cota $N-x>0 $

	\item Veamos la actualización de r en x+1 \par
	      $\quad (\# i,j| 0\leq i < j < x+1 :S[i] \leq 0 \land S[j] \geq 0)$ \par
	      $\equiv \quad \langle$ Último término $ j = x \rangle$ \par
	      $\quad (\# i,j| 0\leq i < j < x :S[i] \leq 0 \land S[j] \geq 0) + (\# i| 0\leq i < x :S[i] \leq 0 \land S[x] \geq 0) $ \par
	      $\equiv \quad \langle$ P0 $\rangle$ \par
	      $\quad r + (\# i| 0\leq i < x :S[i] \leq 0 \land S[x] \geq 0)$ \par

	      Dado que en el cuerpo del cuantificador se tiene la presencia de la variable x
	      en una expresión lógica, se debe analizar por casos.

	      $\equiv \quad \langle$ Casos $\rangle$ \par
	      \begin{equation}
		      \begin{cases}
			      r                                    & \text{si } S[x] < 0    \\
			      r + (\# i| 0\leq i < x :S[i] \leq 0) & \text{si } s[x] \geq 0
		      \end{cases}
	      \end{equation} \par

	\item Se agrega una nueva variable s. \par
	      $s = (\# i| 0\leq i < x :S[i] \leq 0)$ \par
	      $x=0 \Rightarrow s = (\# i| 0\leq i < 0 :S[i] \leq 0)=0$ \par

	      Por lo tanto, se fortalece el invariante con el predicado siguiente. \par

	      \begin{center}
		      $Q: s = (\# i| 0\leq i < x :S[i] \leq 0)$ \par
	      \end{center}

	      De esta manera, la actualización de r queda.

	      \begin{equation}
		      \begin{cases}
			      r     & \text{si } S[x] < 0    \\
			      r + s & \text{si } s[x] \geq 0
		      \end{cases}
	      \end{equation} \par

	\item Se verifica la actualización de s en x+1. \par
	      $\quad (\# i| 0\leq i < x+1 :S[i] \leq 0)$ \par
	      $\equiv \quad \langle$ Último término con $ i = x \rangle$ \par
	      $\quad (\# i| 0\leq i < x :S[i] \leq 0) + (\# i| i = x :S[i] \leq 0)$ \par
	      $\equiv \quad \langle$ Por casos $\rangle$ \par
	      \begin{equation}
		      \begin{cases}
			      (\# i| 0\leq i < x :S[i] \leq 0)     & \text{si } S[x] > 0    \\
			      (\# i| 0\leq i < x :S[i] \leq 0) + 1 & \text{si } S[x] \leq 0
		      \end{cases}
	      \end{equation} \par
	      $\equiv \quad \langle$ Q $ \rangle$ \par
	      \begin{equation}
		      \begin{cases}
			      s     & \text{si } S[x] > 0    \\
			      s + 1 & \text{si } s[x] \leq 0
		      \end{cases}
	      \end{equation} \par

	      Finalmente la actualización de s está dada por. \par
	      \begin{equation}
		      s =
		      \begin{cases}
			      s     & \text{si } S[x] > 0    \\
			      s + 1 & \text{si } s[x] \leq 0
		      \end{cases}
	      \end{equation} \par

\end{enumerate}

\newpage

Programa final. \par

[\par
$\quad const $\par
$\qquad N: int;$\par
$\qquad S: array[0..N) \; of \; int;$\par
$\quad var \; r, s, x: int$\par
$\quad\{N \geq 0\}$\par
$\quad r, s, x := 0,0,0;$\par
$\quad\{inv: r = (\# i,j| 0\leq i < j < x :S[i] \leq 0 \land S[j] \geq 0) \land s = (\# i| 0\leq i < x :S[i] \leq 0) \land 0 \leq x \leq N\}\{cota: N-x\}$\par
$\quad do \; x < N \rightarrow $ \par
$\qquad if \; S[x] < 0 \rightarrow $ \par
$\qquad \quad r = r; $ // Es posible eliminar \par
$\qquad \quad s = s+1; $ \par
$\qquad [\;] \; S[x] > 0 \rightarrow $ \par
$\qquad \quad r = r+s; $ \par
$\qquad \quad s = s; $ // Es posible eliminar \par
$\qquad [\;] \; S[x] = 0 \rightarrow $ \par
$\qquad \quad r = r+s; $ \par
$\qquad \quad s = s+1; $ \par
$\qquad fi $ \par
$\quad od $ \par
$\quad \{r = (\# i,j| 0\leq i < j < N :S[i] \leq 0 \land S[j] \geq 0)\}$ \par
]

\newpage

\section{Matríces}

\subsection{Ejercicios Prof. Chang}

% ************* Start Exercise ******************

\subsubsection{Ejercicio 5: Iteración en matrices. Recorridos}

Dada una matriz A de números enteros, de dimensiones [0..M) x [0..N), se desea
numerar cada casilla comenzando en A[M - 1][0] y comenzando con el número 0
(A[M - 1][0] := 0), siguiendo estrictamente el recorrido que se muestra a
continuación (tomando como ejemplo una matriz 5 x 6).

$A=\begin{bmatrix}
		4 & 5 & 14 & 15 & 24 & 25 \\
		3 & 6 & 13 & 16 & 23 & 26 \\
		2 & 7 & 12 & 17 & 22 & 27 \\
		1 & 8 & 11 & 18 & 21 & 28 \\
		0 & 9 & 10 & 19 & 20 & 29
	\end{bmatrix}$

\vspace{2em}

\textbf{Solución} \par

[ \par
$\quad const \; N, M : int$; \par
$\quad var \; A : array \; [0..M)x[0..N) \; of \; int$; \par
			$\quad var \; i,j, enum : int$; \par
			$\quad i, j, enum := M, 0, 0 $ \par
			$\quad do \; j < N \rightarrow $ \par
			$\qquad i := i + (-1)**(j+1) $ \par
			$\qquad do \; i \; != -1 \land i \; != M \rightarrow $ \par
			$\qquad \quad A[i][j] := enum; $ \par
			$\qquad \quad enum := enum + 1; $ \par
			$\qquad \quad i := i + (-1)**(j+1) $ \par
			$\qquad od $ \par
			$\qquad j := j + 1$ \par
			$\quad od$ \par
		] \par

	\newpage

	En caso de no poder usar la potenciación. \par

	[ \par
$\quad const \; N, M : int$; \par
$\quad var \; A : array \; [0..M)x[0..N) \; of \; int$; \par
		$\quad var \; i,j, enum : int$; \par
		$\quad j, enum := 0, 0 $ \par
		$\quad do \; j < N \rightarrow $ \par
		$\qquad if \; j \mod 2 = 0 \rightarrow $ \par
		$\qquad \quad i = M-1 $ \par
		$\qquad \quad do \; i >= 0 \rightarrow $ \par
		$\qquad \quad \quad A[i][j] := enum; $ \par
		$\qquad \quad \quad enum := enum + 1; $ \par
		$\qquad \quad \quad i := i - 1 $ \par
		$\qquad \quad od $ \par
		$\qquad [\;] \; j \mod 2 \; != 0 \rightarrow $ \par
		$\qquad \quad i = 0 $ \par
		$\qquad \quad do \; i < M \rightarrow $ \par
		$\qquad \quad \quad A[i][j] := enum; $ \par
		$\qquad \quad \quad enum := enum + 1; $ \par
		$\qquad \quad \quad i := i + 1 $ \par
		$\qquad \quad od $ \par
		$\qquad if $; \par
		$\qquad j = j + 1 $; \par
		$\quad od$ \par
	] \par

\newpage

\subsubsection{Ejercicio 6: Recorrido en Arreglos y Matríces, Procedimientos y Funciones.}

% ************* Start Exercise ******************

\textbf{Parte a. Procedimiento:}

Programe correctamente el procedimiento: \par
$move\_in\_The\_Matrix(in-out\;i,\;j:\;int\;,\;in\;N,\;M:\;int,\;in\;TM:\;array\;[0..N)x[0..M)\;of\;int,\;in\;p:\;bool)$ \par
tal que, posicionados en la fila i y en la columna j $(con \; N, M > 0; 0 \leq
	i < N y 0 \leq j < M )$ de la matriz TM, se determine una nueva posición en la
matriz, de la siguiente manera: \par
Si p es true, cuando $TM \; [i - 1][j] \leq TM [i][j]$ nos moveremos una fila
hacia arriba en la matriz. En caso que $TM [i - 1][j] > T M [i][j]$, nos
moveremos una fila hacia abajo. \par
Si p es false, cuando $TM [i][j - 1] \leq T M [i][j]$ nos moveremos una columna
hacia la izquierda en la matriz. En caso que $TM [i][j - 1] > T M [i][j]$ nos
moveremos una columna a la derecha. \par
La matriz es circular, esto es, si i = 0, se considera que $i - 1 = N - 1$ y si
$i = N - 1$, entonces $i + 1 = 0$. De forma similar, si j = 0, se considera que
$j - 1 = M - 1$ y si $j = M - 1$, entonces $j + 1 = 0$. De esta manera, no es
posible desplazarse a posiciones inexistentes de la matriz. Claramente,
actualizar correctamente los valores de i y j es parte de lo que se debe
programar en el procedimiento. En resumen, el procedimiento actulizará el valor
de i o de j, dependiendo del valor de p y de ciertos valores de T M. \par

\newpage

\subsection{Ejercicios Tarea 3 Sep-Dic 2013}

\subsubsection{Ejercicio 15}

% ************* Start Exercise ******************

Mostrar la correctitud.\par
$\{ A[i] = X \land A[j] = Y\} A[i] := A[i] + A[j] \{ A[i] = X + Y\}$\par

\textbf{Solución}\par
Se debe probar la asignación en arreglos.\par
$[P \Rightarrow p \leq i < q \land Q(A := A(i:(A[i]+A[j])))]$\par
Sustituyendo P y Q.\par
$A[i] = X \land A[j] = Y \Rightarrow p \leq i < q \land p \leq j < q \land (A[i] = X + Y)(A := A(i:(A[i]+A[j])))$\par

Por suposición del antecedente empezando por el consecuente.\par

$\quad p \leq i < q \land p \leq j < q \land (A[i] = X + Y)(A := A(i:(A[i]+A[j])))$\par
$\equiv \quad \langle$ Sustitución textual $ \rangle$\par
$\quad p \leq i < q \land p \leq j < q \land A(i:(A[i]+A[j]))[i] = X + Y$\par
$\equiv \quad \langle$ Asignación de Arreglo $ i = j \Rightarrow a[i:E][j] = E \rangle$\par
$\quad p \leq i < q \land p \leq j < q \land A[i]+A[j] = X + Y$\par
$\equiv \quad \langle$ Hipótesis $ A[i] = X \land A[j] = Y \equiv true \Rightarrow p \leq i < q \land p \leq j < q \rangle$\par
$\quad true \land true \land A[i]+A[j] = X + Y$\par
$\equiv \quad \langle$ Hipótesis $ A[i] = X \land A[j] = Y \; Sustitucion \rangle$\par
$\quad  X + Y = X + Y$\par
$\equiv \quad \langle q = q = true \rangle $\par
$\quad  true$\par
$\blacksquare$\par

\newpage

\subsubsection{Ejercicio 16}

% ************* Start Exercise ******************

Mostrar la correctitud.\par
$\{ (\forall i : 0 \leq i < k : A[i] = 2^i) \land 0 \leq k \leq N \land k \neq N\}$\par
$\quad A[k] := 2^k$\par
$\{ (\forall i : 0 \leq i < k : A[i] = 2^i) \land 0 \leq k \leq N\}$\par

\textbf{Solución}\par

Se debe demostrar por regla de la asignación de arreglos.\par
$P \Rightarrow 0 \leq k < N \land Q(A := A(k : 2^k))$\par
Sustituyendo los predicados P y Q.\par
$(\forall i : 0 \leq i < k : A[i] = 2^i) \land 0 \leq k \leq N \land k \neq N \Rightarrow 0 \leq k < N \land ((\forall i : 0 \leq i < k : A[i] = 2^i) \land 0 \leq k \leq N)(A := A(k : 2^k))$\par

Por suposición del antecedente y empezando por el consecuente para alcanzar
true.\par
$\quad 0 \leq k < N \land ((\forall i : 0 \leq i < k : A[i] = 2^i) \land 0 \leq k \leq N)(A := A(k : 2^k))$\par
$\equiv \quad \langle$ Hipótesis $ 0 \leq k \leq N \land k \neq N \equiv 0 \leq k < N \equiv true \rangle$\par
$\quad true \land ((\forall i : 0 \leq i < k : A[i] = 2^i) \land 0 \leq k \leq N)(A := A(k : 2^k))$\par
$\equiv \quad \langle$ Sustitución Textual $ \rangle$\par
$\quad (\forall i : 0 \leq i < k : A(k : 2^k)[i] = 2^i) \land 0 \leq k \leq N$\par
$\equiv \quad \langle$ Definición Asignación de arreglos $ k \neq i \Rightarrow A(k : 2^k)[i] = A[i] \rangle$\par
$\quad (\forall i : 0 \leq i < k : A[i] = 2^i) \land 0 \leq k \leq N$\par
$\equiv \quad \langle$ Hipótesis $ (\forall i : 0 \leq i < k : A[i] = 2^i) \land 0 \leq k \leq N \rangle$\par
$\quad true $\par
$\blacksquare$\par

\newpage

\subsubsection{Ejercicio 17}

% ************* Start Exercise ******************

Mostrar la correctitud.\par
$\{ (\forall i : 0 \leq i < k : S[i] = V[i] \cdot V[i]) \land 0 \leq k \leq N \land k \neq N\}$\par
$\quad S[k],k := V[k] \cdot V[k], k+1$\par
$\{ (\forall i : 0 \leq i < k : S[i] = V[i] \cdot V[i]) \land 0 \leq k \leq N\}$\par

\textbf{Solución}\par

Aplicando la regla de correctitud para la asignación de arreglos, se tiene.\par
$P \Rightarrow 0 \leq k < N \land Q(S, k := S(k : V[k] \cdot V[k]), k+1)$\par

Sustituyendo los predicados P y Q.\par
$(\forall i : 0 \leq i < k : S[i] = V[i] \cdot V[i]) \land 0 \leq k \leq N \land k \neq N \Rightarrow$\par
$0 \leq k < N \land ((\forall i : 0 \leq i < k : S[i] = V[i] \cdot V[i]) \land 0 \leq k \leq N)(S, k := S(k : V[k] \cdot V[k]), k+1)$\par

Sustitución del antecedente, empezando por el consecuente para alcanzar true.\par
$\quad 0 \leq k < N \land ((\forall i : 0 \leq i < k : S[i] = V[i] \cdot V[i]) \land 0 \leq k \leq N)(S, k := A(k : V[k] \cdot V[k]), k+1)$\par
$\equiv \quad \langle$ Hipótesis $ 0 \leq k \leq N \land k \neq N \equiv 0 \leq k < N \equiv true \rangle$\par
$\quad true \land ((\forall i : 0 \leq i < k : S[i] = V[i] \cdot V[i]) \land 0 \leq k \leq N)(S, k := S(k : V[k] \cdot V[k]), k+1)$\par
$\equiv \quad \langle$ Sustitución Textual $ \rangle$\par
$\quad (\forall i : 0 \leq i < k+1 : S(k : V[k] \cdot V[k])[i] = V[i] \cdot V[i]) \land 0 \leq k+1 \leq N$\par
$\equiv \quad \langle$ Sacando último término $ i = k \rangle$\par
$\quad (\forall i : 0 \leq i < k : S(k : V[k] \cdot V[k])[i] = V[i] \cdot V[i]) \land S(k : V[k] \cdot V[k])[k] = V[k] \cdot V[ik] \land 0 \leq k+1 \leq N$\par
$\equiv \quad \langle$ Definición Asignación de Arreglos ${ i = k \Rightarrow a[k : F][i] = F \land  i \neq k \Rightarrow a[k : F][i] = a[i] }\rangle$\par
$\quad (\forall i : 0 \leq i < k : S[i] = V[i] \cdot V[i]) \land V[k] \cdot V[k] = V[k] \cdot V[k] \land 0 \leq k+1 \leq N$\par
$\equiv \quad \langle$ Hipótesis $(\forall i : 0 \leq i < k : S[i] = V[i] \cdot V[i]); \; q=q=true\rangle$\par
$\quad true \land true \land 0 \leq k+1 \leq N$\par
$\Leftarrow \quad \langle k \geq 0 \Rightarrow k+1 \geq 0; \; k \leq N \land k \neq N \equiv k < N \Rightarrow k + 1 \leq N \rangle$\par
$\quad 0 \leq k \leq N \land k \neq N$\par
$\equiv \quad \langle$ Hipótesis $0 \leq k \leq N \land k \neq N\rangle$\par
$\quad true $\par
$\blacksquare$\par

\newpage

\section{Procedimientos y Funciones}

\subsection{Caso general B2 (Teoría)}

$\neg ocurreLibre('a,b', E) \equiv false$ \par
$proc \; p \; (entrada \; x; \; entrada-salida \; y; \; salida \; z) $ \par
$\quad \{P_{def}\}$ \par
$\quad \{Q_{def}\}$ \par
$\quad S$ \par

Llamada.\par

$\quad \{P_{llam}\} \; p(E,a,b) \; \{Q_{llam}\}$ \par

Demostraciones:
\begin{enumerate}
	\item $[P_{llam} \Rightarrow P_{def}(x,y := E,a)]$ \par
	\item $[P_{llam}(a,b := A,B) \land Q_{def}(x,y_0,y,z := E(a,b := A,B),A,a,b) \Rightarrow Q_{llam}]$ \par
\end{enumerate}

\newpage

% ************* Start Exercise ******************

\subsection{Ejercicio Practica Ene-Mar 2025}

Dado el procedimiento. \par
$proc \; sumar \; (entrada \; i,d,N:int; entrada-salida \; a: \; array[0..N) \; de \; int) $ \par
$\quad \{P_{def}: d>0 \land N>0 \land 0 \leq i < N\}$ \par
$\quad \{Q_{def}: a[i] = a_0[i] + d\}$ \par

Demostrar la correctitud. \par

$\{P_{llam}: a \geq M \geq 73 \land p[0]=0\}$ \par
$\quad sumar(p[0],a,M,p)$ \par
$\{Q_{llam}: p[0] \geq 73\}$ \par

\vspace{2em}

\textbf{Demostración:}

Entradas: $p[0](p), \; a, \; M$ \par
Entradas-salidas: $p$ \par

Se debe demostrar (Definición): \par
\begin{enumerate}
	\item $[P_{llam} \Rightarrow P_{def}(x,y := E,a)]$ \par
	\item $[P_{llam}(a,b := A,B) \land Q_{def}(x,y_0,y,z := E(a,b := A,B),A,a,b) \Rightarrow Q_{llam}]$ \par
\end{enumerate}

Donde: \par
($\;'parametros' \rightarrow \; 'argumentos'$) \par
$x= \;'i,d,N' \rightarrow \; 'p[0], a, M'$ \par
$y= \;'a' \rightarrow \; 'p'$ \par
$z=''$ \par

Las sustituciones a realizar: \par

\begin{enumerate}
	\item $[P_{llam} \Rightarrow P_{def}(i,d,N,a := p[0],a,M,p)]$ \par
	\item $[P_{llam}(p := P) \land Q_{def}(i,d,N,a_0,a := (p[0])(p := P),a, M, P, p) \Rightarrow Q_{llam}]$ \par
\end{enumerate}

Demostraando las expresiones: \par

\begin{enumerate}
	\item $[P_{llam} \Rightarrow P_{def}(i,d,N,a := p[0],a,M,p)]$ \par
	      $a \geq M \geq 73 \land p[0]=0 \Rightarrow (d>0 \land N>0 \land 0 \leq i < N)(i,d,N,a := p[0],a,M,p)$ \par
	      Suposición del antecedente empezando por consecuente para alcanzar true. \par
	      $\qquad (d>0 \land N>0 \land 0 \leq i < N)(i,d,N,a := p[0],a,M,p)$ \par
	      $\equiv \qquad \langle S.T \rangle$ \par
	      $\qquad a>0 \land M>0 \land 0 \leq p[0] < M$ \par
	      $\equiv \qquad \langle$ Hipótesis   $p[0]=0\rangle$ \par
	      $\qquad a>0 \land M>0 \land 0 \leq 0 < M$ \par
	      $\equiv \qquad \langle$ Hipótesis   $M \geq 73 \Rightarrow M > 0 \equiv true \land a \geq M \geq 73 \Rightarrow a > 0 \equiv true \rangle$ \par
	      $\qquad true \land true \land 0 \leq 0 < M$ \par
	      $\equiv \qquad \langle$ Aritmética   $0 \leq 0 < M \equiv 0 \leq 0 \land 0 < M \rangle$ \par
	      $\qquad 0 \leq 0 \land 0 < M$ \par
	      $\equiv \qquad \langle$ Aritmética   $0 \leq 0 \equiv true < M \rangle$ \par
	      $\qquad 0 < M$ \par
	      $\equiv \qquad \langle$ Hipótesis   $M \geq 73 \Rightarrow M > 0 \equiv true \rangle$ \par
	      $\qquad true$ \par
	\item $[P_{llam}(p := P) \land Q_{def}(i,d,N,a_0,a := (p[0])(p := P),a, M, P, p) \Rightarrow Q_{llam}]$ \par
	      $(a \geq M \geq 73 \land p[0]=0)(p := P) \land Q_{def}(i,d,N,a_0,a := (p[0])(p := P),a, M, P, p) \Rightarrow p[0] \geq 73$ \par
	      Debilitamiento. \par
	      $\qquad (a \geq M \geq 73 \land p[0]=0)(p := P) \land (a[i] = a_0[i] + d)(i,d,N,a_0,a := (p[0])(p := P),a, M, P, p)$ \par
	      $\equiv \qquad \langle S.T \rangle$ \par
	      $\qquad a \geq M \geq 73 \land P[0]=0 \land p[P[0]] = P[P[0]] + a$ \par
	      $\equiv \qquad \langle$ Sustitución $ P[0]=0 \rangle$ \par
	      $\qquad a \geq M \geq 73 \land P[0]=0 \land p[0] = P[0] + a$ \par
	      $\equiv \qquad \langle$ Sustitución $ P[0]=0 \rangle$ \par
	      $\qquad a \geq M \geq 73 \land P[0]=0 \land p[0] = 0 + a$ \par
	      $\equiv \qquad \langle$ Sustitución $ p[0]=a \rangle$ \par
	      $\qquad p[0] \geq M \geq 73 \land P[0]=0 \land p[0] = a$ \par
	      $\Rightarrow \qquad \langle$ Transitividad $ a \geq b \geq c \Rightarrow a \geq c \rangle$ \par
	      $\qquad p[0] \geq 73 \land P[0]=0 \land p[0] = a$ \par
	      $\Rightarrow \qquad \langle$ Debilitamiento $ p \land q \Rightarrow p \rangle$ \par
	      $\qquad p[0] \geq 73 $ \par
		  $\blacksquare$
\end{enumerate}

\newpage

% ************* Start Exercise ******************

\subsubsection{Ejercicio Examen Ene-Mar 2025}

Dado el procedimiento.\par
$proc \; intercambiar \; \; (entrada \; i,N:int; \; entrada-salida \; a,b: \; array[0..N) \; de \; int) $\par
$\quad \{Pre: N > 0 \land 0 \leq i < N\}$\par
$\quad \{Post: a[i] = b_0[i] \land b[i] = a_0[i]\}$\par

Demostrar la correctitud.\par

$\{M > 5 \land p[0] > -1 \land a[0] > 3 \land p[0] + a[0] = X \land X < 3 \land a[x] = C\}$\par
$\quad intercambiar(p[0]+a[0],M,p,a)$\par
$\{p[X] = C\}$ \par

\vspace{2em}

\textbf{Demostración:}\par

$1. [P_{llam} \Rightarrow P_{def}(x,y := E,a)]$\par
$2.m[P_{llam}(a,b := A,B) \land Q_{def}(x,y_0,y,z := E,A,a,b) \Rightarrow Q_{llam}]$\par

\hspace{4em} $parameter \rightarrow argumentos$\par

Entradas: $i,N \rightarrow p[0] + a[0],M$\par
Entrada-Salida: $a,b \rightarrow p,a$\par
Valores iniciales: $b_0, a_0$\par

Las pruebas a realizar quedan:\par
$1. [P_{llam} \Rightarrow P_{def}(i,N,a,b := p[0] + a[0],M, p,a)]$\par
$2. [P_{llam}(p,a := P,A) \land Q_{def}(i,N,a_0,b_0,a,b := (p[0] + a[0])(p,a := P,A),M,P, A, p, a) \Rightarrow Q_{llam}]$\par

Sustituyendo los predicados.\par
$1. [M > 5 \land p[0] > -1 \land a[0] > 3 \land p[0] + a[0] = X \land X < 3 \land a[x] = C \Rightarrow \\
 (N > 0 \land 0 \leq i < N)(i,N,a,b := p[0] + a[0],M, p,a)]$\par
$2. [(M > 5 \land p[0] > -1 \land a[0] > 3 \land p[0] + a[0] = X \land X < 3 \land a[x] = C)(p,a := P,A) \land (a[i] = b_0[i] \land b[i] = a_0[i])(i,N,a_0,b_0,a,b := (p[0] + a[0])(p,a := P,A),M,P, A, p, a) \Rightarrow p[X] = C]$\par

Demostración 1.\par
$M > 5 \land p[0] > -1 \land a[0] > 3 \land p[0] + a[0] = X \land X < 3 \land a[x] = C \Rightarrow \\
 (N > 0 \land 0 \leq i < N)(i,N,a,b := p[0] + a[0],M, p,a)$\par
 Aplicando sustitución textual.\par
 $M > 5 \land p[0] > -1 \land a[0] > 3 \land p[0] + a[0] = X \land X < 3 \land a[x] = C \Rightarrow M > 0 \land 0 \leq p[0] + a[0] < M$\par

Por debilitamiento.\par
$\qquad M > 5 \land p[0] > -1 \land a[0] > 3 \land p[0] + a[0] = X \land X < 3 \land a[x] = C$\par
$\Rightarrow \qquad \langle a > b \land c > d \Rightarrow a + c > b + d \rangle$\par
$\qquad M > 5 \land p[0] + a[0] > 2 \land p[0] + a[0] = X \land X < 3 \land a[x] = C$\par
$\equiv \qquad \langle $Sustitución$ p[0] + a[0] = X \rangle$\par
$\qquad M > 5 \land X > 2 \land p[0] + a[0] = X \land X < 3 \land a[x] = C$\par
$\Rightarrow \qquad \langle $Debilitamiento$ p \land q \Rightarrow p \rangle$\par
$\qquad X > 2 \land X < 3 $\par
$\equiv \qquad \langle $Rango vacío$ \rangle$\par
$\qquad false $\par
$\equiv \qquad \langle false \Rightarrow p \equiv true \rangle$\par
$\qquad M > 0 \land 0 \leq p[0] + a[0] < M $\par
$\blacksquare $\par

Demostración 2.\par
$(M > 5 \land p[0] > -1 \land a[0] > 3 \land p[0] + a[0] = X \land X < 3 \land a[X] = C)(p,a := P,A) \land (a[i] = b_0[i] \land b[i] = a_0[i])(i,N,a_0,b_0,a,b := (p[0] + a[0])(p,a := P,A),M,P, A, p, a) \Rightarrow p[X] = C$\par

Aplicando debilitamiento.\par
$\qquad (M > 5 \land p[0] > -1 \land a[0] > 3 \land p[0] + a[0] = X \land X < 3 \land a[X] = C)(p,a := P,A) \land (a[i] = b_0[i] \land b[i] = a_0[i])(i,N,a_0,b_0,a,b := (p[0] + a[0])(p,a := P,A),M,P, A, p, a)$\par
$\equiv \qquad \langle$ S.T $p,a := P,A\rangle$\par
$\qquad M > 5 \land P[0] > -1 \land A[0] > 3 \land P[0] + A[0] = X \land X < 3 \land A[X] = C \land (a[i] = b_0[i] \land b[i] = a_0[i])(i,N,a_0,b_0,a,b := (p[0] + a[0])(p,a := P,A),M,P, A, p, a)$\par
$\equiv \qquad \langle$ S.T $\rangle$\par
$\qquad M > 5 \land P[0] > -1 \land A[0] > 3 \land P[0] + A[0] = X \land X < 3 \land A[X] = C \land p[P[0] + A[0]] = A[P[0] + A[0]] \land a[P[0] + A[0]] = P[P[0] + A[0]]$\par
$\equiv \qquad \langle$ S.T $P[0] + A[0] = X\rangle$\par
$\qquad M > 5 \land P[0] > -1 \land A[0] > 3 \land P[0] + A[0] = X \land X < 3 \land A[X] = C \land p[X] = A[X] \land a[X] = P[X]$\par
$\equiv \qquad \langle$ S.T $A[X] = C\rangle$\par
$\qquad M > 5 \land P[0] > -1 \land A[0] > 3 \land P[0] + A[0] = X \land X < 3 \land A[X] = C \land p[X] = C \land a[X] = P[X]$\par
$\Rightarrow \qquad \langle$ Debilitamiento $p \land q \Rightarrow p\rangle$\par
$\qquad p[X] = C $\par
$\blacksquare $\par

\newpage

% ************* Start Exercise ******************

\subsection{Ejercicios Prof. Chang}

\subsubsection{Ejercicio 7: Corrección de llamadas a procedimientos. Euclides}

Dado el siguiente procedimiento: \par

proc euclides(in b, c :int ; out q,r :int) \par
$\quad \{P: b \geq 0 \land c > 0\}$ \par
$\quad \{Q: b=q \cdot c + r \land 0 \leq r < c\}$ \par

Tenga en cuenta que los operadores div y mod se definen en base a esta
descomposición de b: \par

\begin{center}
	b div c = q , y b mod c = r
\end{center}

Si la siguiente tripleta de Hoare es correcta, demuéstrela formalmente. En caso
de ser incorrecta, explique por qué:\par

$\{ q = 5 \land 0 \leq r < 3 \} \; euclides(q + r, q - r, b, c) \; \{ (b = 1 \lor b = 2) \land 0 \leq c < 3 \}$ \par

\textbf{Solución}

Se usa el caso general B2. Por lo que se debe demostrar lo siguiente. \par

$1. [P_{llam} \Rightarrow P_{def}(x,y := E, a)]$ \par

$2. [P_{llam}(a,b := A, B) \land Q_{def}(x,y_0,y,z := E(a,b := A,B), A, a,b) \Rightarrow Q_{llam}]$ \par

Para este problema particular se tienen las siguientes variables: \par
\hspace{6em} $parametros \rightarrow argumentos$ \par
Entradas (x): $b,c \rightarrow (q+r), (q-r)$ \par
Salidas (z): $q,r \rightarrow b, c$ \par

Se debe probar: \par

$1. [P_{llam} \Rightarrow P_{def}(b,c := q+r, q-r)]$ \par

$2. [P_{llam} \land Q_{def}(b,c,q,r := q+r,q-r, b, c) \Rightarrow Q_{llam}]$ \par

Prueba 1. \par

$q=5 \land 0 \leq r < 3 \Rightarrow q+r \geq 0 \land q-r > 0$ \par

Aplicando método de suposición del antecedente. \par

$\quad true$ \par
$\equiv \quad \langle$ Hipótesis $ 0 \leq r < 3 \rangle$ \par
$\quad 0 \leq r < 3$ \par
$\equiv \quad \langle$ Aritmética $ \rangle$ \par
$\quad 0 \leq r \land r < 3$ \par
$\equiv \quad \langle$ Aritmética sumar 5 a ambos lados de $ 0 \leq r \rangle$ \par
$\quad 5 \leq r +5 \land r < 3$ \par
$\equiv \quad \langle$ Aritmética resto 5 a ambos lados de $ r < 3 \rangle$ \par
$\quad 5 \leq r +5 \land r - 5 < 3 - 5$ \par
$\equiv \quad \langle$ Aritmética $ \rangle$ \par
$\quad 5 \leq r +5 \land r - 5 < -2$ \par
$\equiv \quad \langle$ Aritmética: multiplicar -1 a ambos lados de $r - 5 < -2 \rangle$ \par
$\quad 5 \leq r +5 \land 5 - r > 2$ \par
$\Rightarrow \quad \langle$ Hipótesis y sustitución Leibniz $ q = 5 \rangle$ \par
$\quad 5 \leq r + q \land q - r > 2$ \par
$\Rightarrow \quad \langle 5 \leq a \Rightarrow 0 \leq a \rangle$ \par
$\quad 0 \leq r + q \land q - r > 2$ \par
$\Rightarrow \quad \langle a > 2 \Rightarrow a > 0 \rangle$ \par
$\quad 0 \leq r + q \land q - r > 0$ \par
$\equiv \quad \langle$ Simetría $ a > b \equiv b < a \rangle$ \par
$\quad r + q \geq 0 \land q - r > 0$ \par

$\blacksquare$ \par

Prueba 2. \par

$q=5 \land 0 \leq r < 3 \land q + r = b \cdot (q - r) + c \land 0 \leq c < q - r \Rightarrow (b = 1 \lor b = 2) \land 0 \leq c < 3$ \par

\textbf{Contraejemplo} \par

Hipótesis: \par
$q=5 \land r=1 \land 0 \leq c < q-r \Rightarrow q=5 \land r=1 \land 0 \leq c < 5-1 \Rightarrow q=5 \land r=1 \land 0 \leq c < 4 \Rightarrow q=5 \land r=1 \land c=3$ \par

Tomemos la siguiente hipótesis para sustituir los valores particulares de
$q,r,c$: \par
$q + r = b \cdot (q - r) + c \Rightarrow 5 + 1 = b \cdot (5 - 1) + 3 \equiv 6 = b \cdot 4 + 3 \equiv 3 = b \cdot 4 \Rightarrow b = \frac{3}{4}$ \par

Lo cual no corresponde con el consecuente que indica que b es 1 o 2, de igual
manera indica que $c < 3$. \par

\newpage

% ************* Start Exercise ******************

\subsubsection{Ejercicio 8: Corrección de llamadas a procedimientos. Productos Notables}

Dado el siguiente procedimiento: \par

proc productoNotableLimitado(in a, b :int ; out z :int) \par
$\quad \{P: b \geq 0 \land b > a\}$ \par
$\quad \{Q: z = a^2 + 2ab +b^2\}$ \par

Si la siguiente tripleta de Hoare es correcta, demuéstrela formalmente. En caso
de ser incorrecta, explique por qué:\par

$\{ 0 < b < 5 \land z = -1 \land c = 1 \} \; productoNotableLimitado(b + z, b + c, b) \; \{ 2^2 \leq b \leq 2^6 \}$ \par

\textbf{Solución} \par

Se implementará el caso general B2, cuyas pruebas se muestran a continuación. \par

$1. [P_{llam} \Rightarrow P_{def}(x,y := E, a)]$ \par
$2. [P_{llam}(a,b := A,B) \land Q_{def}(x,y_0,y,z := E(a,b := A,B), A, a,b) \Rightarrow Q_{llam}]$ \par

Para este problema se tiene lo siguiente: \par

\hspace{6em} parámetros $\rightarrow$ argumentos \par

Entradas (x): $a,b \rightarrow b+z, b+c$ \par
Salidas (z): $z \rightarrow b$ \par
Valor inicial (out b): $b \rightarrow B$ \par

De esta manera las pruebas a realizar son las siguientes. \par

$1. [P_{llam} \Rightarrow P_{def}(a,b := b+z,b+c)]$ \par
$2. [P_{llam}(b := B) \land Q_{def}(a,b,z := (b+z)(b := B), (b+c)(b := B), b) \Rightarrow Q_{llam}]$ \par

Prueba 1. \par
$ 0 < b < 5 \land z = -1 \land c = 1 \Rightarrow b+c \geq 0 \land b+c > b+z$ \par

Por debilitamiento. \par

$ \quad 0 < b < 5 \land z = -1 \land c = 1$ \par
$\equiv \quad \langle$ Aritmetica $\rangle$ \par
$ \quad 0 < b \land b < 5 \land z = -1 \land c = 1$ \par
$\equiv \quad \langle$ Aritmetica sumando 1 a ambos lados de $ 0 < b \rangle$ \par
$ \quad 1 < b+1 \land b < 5 \land z = -1 \land c = 1$ \par
$\Rightarrow \quad \langle a > 1 \Rightarrow a > 0 \rangle$ \par
$ \quad b+1 > 0 \land b < 5 \land z = -1 \land c = 1$ \par
$\Rightarrow \quad \langle a > 0 \Rightarrow a \geq 0 \land a > 0 \rangle$ \par
$ \quad b+1 \geq 0 \land b+1 > 0 \land b < 5 \land z = -1 \land c = 1$ \par
$\Rightarrow \quad \langle a > 0 \Rightarrow a + 1 > 0 \Rightarrow a + 1 > a - 1  \rangle$ \par
$ \quad b+1 \geq 0 \land b+1 > b - 1 \land b < 5 \land z = -1 \land c = 1$ \par
$\Rightarrow \quad \langle$ Sustitucion Leibniz  $c = 1; z = -1\rangle$ \par
$ \quad b+c \geq 0 \land b+c > b - z$ \par

$\blacksquare$ \par
Prueba 2. \par
$\quad (0 < b < 5 \land z = -1 \land c = 1)(b := B) \land (z = a^2 + 2ab +b^2)(a,b,z := (b+z)(b := B), (b+c)(b := B), b) \Rightarrow 2^2 \leq b \leq 2^6$ \par
$\equiv \quad \langle$ Sustitución $ b := B\rangle$ \par
$\quad 0 < B < 5 \land z = -1 \land c = 1 \land (z = a^2 + 2ab +b^2)(a,b,z := (B+z), (B+c), b) \Rightarrow 2^2 \leq b \leq 2^6$ \par
$\equiv \quad \langle$ Sustitución $ \rangle$ \par
$\quad 0 < B < 5 \land z = -1 \land c = 1 \land b = (B+z)^2 + 2(B+z)(B+c) +(B+c)^2 \Rightarrow 2^2 \leq b \leq 2^6$ \par

Debilitamiento. \par
$\quad 0 < B < 5 \land z = -1 \land c = 1 \land b = (B+z)^2 + 2(B+z)(B+c) +(B+c)^2$ \par
$\Rightarrow \quad \langle$ Sustitución Leibniz $ z = -1; c = 1 \rangle$ \par
$\quad 0 < B < 5 \land b = (B-1)^2 + 2(B-1)(B+1) +(B+1)^2 \Rightarrow 2^2 \leq b \leq 2^6$ \par
$\equiv \quad \langle$ Aritmética $ \rangle$ \par
$\quad 0 < B < 5 \land b = B^2 -2B + 1 + 2B^2 - 2 + B^2 +2B + 1 $ \par
$\equiv \quad \langle$ Aritmética $ \rangle$ \par
$\quad 0 < B < 5 \land b = 4B^2$ \par
$\equiv \quad \langle$ Aritmética $ \rangle$ \par
$\quad 0 < B^2 < 25 \land b = 4B^2$ \par
$\equiv \quad \langle$ Aritmética $ \rangle$ \par
$\quad 0 < 4B^2 < 100 \land b = 4B^2$ \par
$\Rightarrow \quad \langle$ Sustitución Leibniz $ \rangle$ \par
$\quad 0 < b < 100 $ \par
$\equiv \quad \langle$ Aritmética $ \rangle$ \par
$\quad 0 < b \land b < 5 \cdot 2^2 $ \par
$\Rightarrow \quad \langle b > 0 \Rightarrow b \geq 2^2; \; b < 5\cdot 2^2 \Rightarrow b < 2^6 \rangle$ \par
$\quad 2^2 \leq b \land b < 2^6 $ \par
$\equiv \quad \langle a < b \equiv a \leq b \land a \neq b \rangle$ \par
$\quad 2^2 \leq b \land b \leq 2^6 \land b \neq 2^6 $ \par
$\Rightarrow \quad \langle p \land q \Rightarrow p \rangle$ \par
$\quad 2^2 \leq b \land b \leq 2^6 $ \par
$\equiv \quad \langle$ Aritmética $ \rangle$ \par
$\quad 2^2 \leq b \leq 2^6 $ \par
$\blacksquare$ \par

\newpage

\subsection{Ejercicios Tarea 3 Sep-Dic 2013}

\subsubsection{Ejercicio 12}

% ************* Start Exercise ******************

Mostrar la correctitud.\par

$\quad \{x > 0\}$ \par
$\quad P(x)$ \par
$\{ x \mod 2 = 1\}$ \par

$\textbf{proc} \; P(in-out \; a: entero)$ \par
$\quad \{pre: a \geq 0\}$ \par
$\quad \{post: a = 2a_0+1\}$\par

\textbf{Solución}\par

Vamos a aplicar el caso general B2, para lo cual se debe demostrar lo
siguiente.\par

$1. [P_{llam} \Rightarrow P_{def}(a := x)]$\par
$2. [P_{llam}(x := X) \land Q_{def}(a_0,a := X, x) \Rightarrow Q_{llam}]$\par

\begin{absolutelynopagebreak}
	Prueba 1.\par
	$x > 0 \Rightarrow x \geq 0$\par

	Fortalecimiento.\par
	$\quad x \geq 0$\par
	$\equiv \quad \langle a \geq b \equiv a > b \lor a = b \rangle$\par
	$\quad x > 0 \lor x = 0$\par
	$\Leftarrow \quad \langle p \Rightarrow p \lor q \rangle$\par
	$\quad x > 0$\par
	$\blacksquare$\par
\end{absolutelynopagebreak}

\begin{absolutelynopagebreak}
	Prueba 2.\par
	$(x > 0)(x := X) \land (a = 2a_0 + 1)(a_0, a := X, x) \Rightarrow x \mod 2 = 1$\par

	Debilitamiento.\par
	$\quad (x > 0)(x := X) \land (a = 2a_0 + 1)(a_0, a := X, x)$\par
	$\equiv \quad \langle S.T \rangle$\par
	$\quad X > 0 \land x = 2X + 1$\par
	$\equiv \quad \langle a \mod 2 = 1 \equiv \exists \; b > 0 | a = 2b + 1  \rangle$\par
	$\quad X > 0 \land x \mod 2 = 1$\par
	$\Rightarrow \quad \langle p \land q \Rightarrow p  \rangle$\par
	$\quad x \mod 2 = 1$\par
	$\blacksquare$\par
\end{absolutelynopagebreak}

\newpage

\subsubsection{Ejercicio 13}

% ************* Start Exercise ******************

Mostrar la correctitud.\par

$\quad \{n \geq m \land m \geq 0\}$ \par
$\quad Pfactorial(n,c)$ \par
$\{ c = (\prod i| 1 \leq i \leq n : i)\}$ \par

$\textbf{proc} \; Pfactorial(in \; x: entero; \; out \; f: entero)$ \par
$\quad \{pre: x \geq 0\}$ \par
$\quad \{post: f = (\prod i| 1 \leq i \leq x: i)\}$\par

\textbf{Solución}\par

Vamos a usar el caso general B2, para lo cual se debe demostrar lo siguiente.\par
$1. [P_{llam} \Rightarrow P_{def}(x := n)]$\par
$2. [P_{llam}(c := C) \land Q_{def}(x,f := n,c) \Rightarrow Q_{llam}]$\par

\begin{absolutelynopagebreak}
	Prueba 1.\par
	$n \geq m \land m \geq 0 \Rightarrow n \geq 0$\par

	Debilitamiento.\par
	$\quad n \geq m \land m \geq 0$\par
	$\Rightarrow \quad \langle Transitividad \rangle$\par
	$\quad n \geq  0$\par
	$\blacksquare$\par
\end{absolutelynopagebreak}

\begin{absolutelynopagebreak}
	Prueba 2.\par
	$n \geq m \land m \geq 0 \land (f = (\prod i| 1 \leq i \leq x: i))(x,f := n,c) \Rightarrow c = (\prod i| 1 \leq i \leq n : i)$\par

	Debilitamiento.\par
	$\quad n \geq m \land m \geq 0 \land (f = (\prod i| 1 \leq i \leq x: i))(x,f := n,c)$\par
	$\equiv \quad \langle S.T \rangle$\par
	$\quad n \geq m \land m \geq 0 \land c = (\prod i| 1 \leq i \leq n: i)$\par
	$\Rightarrow \quad \langle p \land q \Rightarrow p \rangle$\par
	$\quad c = (\prod i| 1 \leq i \leq n: i)$\par
	$\blacksquare$\par
\end{absolutelynopagebreak}

\newpage

\subsubsection{Ejercicio 14}

% ************* Start Exercise ******************

Mostrar la correctitud de la siguiente terna de Hoare.\par

$\{0 \leq k \leq N \land s = (\sum i| 0 \leq i < k: (\prod i| 0 \leq i \leq k : x^i) / (\prod i| 1 \leq i \leq k : i))\}$ \par
$\quad s,k := s + potencia(x,k)/factorial(k), k + 1$ \par
$\{ s = (\sum i| 0 \leq i < k: (\prod i| 0 \leq i \leq k : x^i) / (\prod i| 1 \leq i \leq k : i))\}$ \par

De donde se tiene, \par

$\textbf{func} \; potencia(y: real, n: entero) \rightarrow real$ \par
$\quad \{pre: n \geq 0\}$ \par
$\quad \{post: potencia= (\prod i| 0 \leq i < n: y^i)\}$\par

$\textbf{func} \; factorial(x: entero) \rightarrow entero$ \par
$\quad \{pre: x \geq 0\}$ \par
$\quad \{post: factorial= (\prod i| 1 \leq i \leq x: i)\}$\par

\textbf{Solución}\par

Se definen las siguientes variables para las llamadas de las funciones potencia
y factorial.\par

$\phi: $ Retorno de potencia.\par
$\theta: $ Retorno de factorial.\par

Los procedimientos quedan definidos como siguen.\par
\begin{absolutelynopagebreak}
	$\textbf{proc} \; proc\_potencia(in \; y: real, n: entero; \; out \; p: real)$ \par
	$\quad \{pre: n \geq 0\}$ \par
	$\quad \{post: p = (\prod i| 0 \leq i < n: y^i)\}$\par

	$\textbf{proc} \; proc\_factorial(in \; x: entero; \; out \; f: entero)$ \par
	$\quad \{pre: x \geq 0\}$ \par
	$\quad \{post: f = (\prod i| 1 \leq i \leq x: i)\}$\par
\end{absolutelynopagebreak}

La tripleta nos queda. \par

\begin{absolutelynopagebreak}
	$\{0 \leq k \leq N \land s = (\sum i| 0 \leq i < k: (\prod i| 0 \leq i \leq k : x^i) / (\prod i| 1 \leq i \leq k : i))\}$ \par
	$\quad proc\_potencia(x,k,\phi)$; \par
	$\quad proc\_factorial(k,\theta)$; \par
	$\quad s,k := s + \phi/\theta, k + 1$ \par
	$\{ s = (\sum i| 0 \leq i < k: (\prod i| 0 \leq i \leq k : x^i) / (\prod i| 1 \leq i \leq k : i))\}$ \par
\end{absolutelynopagebreak}

Se determina la precondición más débil.\par

\begin{absolutelynopagebreak}
	$\{0 \leq k \leq N \land s = (\sum i| 0 \leq i < k: (\prod i| 0 \leq i \leq k : x^i) / (\prod i| 1 \leq i \leq k : i))\}$ \par
	$\quad proc\_potencia(x,k,\phi)$; \par
	$\quad proc\_factorial(k,\theta)$; \par
	$\{ s + \phi/\theta = (\sum i| 0 \leq i < k + 1: (\prod i| 0 \leq i \leq k + 1 : x^i) / (\prod i| 1 \leq i \leq k + 1 : i))\}$ \par
	$\quad s,k := s + \phi/\theta, k + 1$ \par
	$\{ s = (\sum i| 0 \leq i < k: (\prod i| 0 \leq i \leq k : x^i) / (\prod i| 1 \leq i \leq k : i))\}$ \par
\end{absolutelynopagebreak}

\newpage

\subsubsection{Ejercicio 18}

% ************* Start Exercise ******************

Mostrar la correctitud.\par
$\{ i \geq 0 \land i < N - 1 \land A[i] = X \land A[i+1] = Y\}$\par
$\quad Intercambio(N,A,i)$\par
$\{ A[i] = Y \land A[i + 1] = X\}$\par

$\textbf{proc} \; Intercambio(in \; M: entero; \; in-out \; A: arreglo \; [0..M) \; de \; enteros; in \; r: entero)$ \par
$\quad \{pre: 0 \leq r < M - 1\}$ \par
$\quad \{post: A[r] = A_0[r + 1] \land A[r + 1] = A_0[r]\}$\par

\textbf{Solución}\par

Aplicando la regla general B2 tenemos lo siguiente.\par

\hspace{5em} $parameters \rightarrow argumentos$\par

In: $M, r \rightarrow N, i$\par
In-out: $A \rightarrow A$. Valor inicial: $A_0$\par

Se debe demostrar lo siguiente.\par
$1. P_{llam} \Rightarrow P_{def}(M,r,A := N, i, A)$\par
$2. P_{llam}(A := AA) \land Q_{def}(M, r, A_0, A := N, i, AA, A) \Rightarrow Q_{llam}$\par
\begin{absolutelynopagebreak}
	Prueba 1.\par
	$i \geq 0 \land i < N - 1 \land A[i] = X \land A[i+1] = Y \Rightarrow (0 \leq r < M - 1)(M,r,A := N, i, A)$\par

	Suposición del antecedente, empezando por el consecuente para alcanzar true.\par
	$\quad (0 \leq r < M - 1)(M,r,A := N, i, A)$\par
	$\equiv \quad \langle S.T \rangle$\par
	$\quad 0 \leq i < N - 1$\par
	$\equiv \quad \langle$ Aritmética $ \rangle$\par
	$\quad 0 \leq i \land i < N - 1$\par
	$\equiv \quad \langle$ Hipótesis $ i \geq 0 \equiv true; \; i < N - 1 \equiv true \rangle$\par
	$\quad true \land true$\par
	$\equiv \quad \langle$ Identidad $ true \land true \equiv true \rangle$\par
	$\quad true$\par
	$\blacksquare$\par
\end{absolutelynopagebreak}

\begin{absolutelynopagebreak}
	Prueba 2.\par
	$(i \geq 0 \land i < N - 1 \land A[i] = X \land A[i+1] = Y)(A := AA) \land $\par
	$ (A[r] = A_0[r + 1] \land A[r + 1] = A_0[r])(M, r, A_0, A := N, i, AA, A) \Rightarrow A[i] = Y \land A[i + 1] = X$\par

	Debilitamiento.\par
	$\quad (i \geq 0 \land i < N - 1 \land A[i] = X \land A[i+1] = Y)(A := AA) \land $\par
	$\qquad (A[r] = A_0[r + 1] \land A[r + 1] = A_0[r])(M, r, A_0, A := N, i, AA, A)$\par
	$\equiv \quad \langle$ Sustitución textual $ A := AA \rangle$\par
	$\quad i \geq 0 \land i < N - 1 \land AA[i] = X \land AA[i+1] = Y \land $\par
	$\qquad (A[r] = A_0[r + 1] \land A[r + 1] = A_0[r])(M, r, A_0, A := N, i, AA, A)$\par
	$\equiv \quad \langle$ Sustitución textual $ \rangle$\par
	$\quad i \geq 0 \land i < N - 1 \land AA[i] = X \land AA[i+1] = Y \land A[i] = AA[i + 1] \land A[i + 1] = AA[i]$\par
	$\equiv \quad \langle$ Sustitución $ AA[i] = X \land AA[i+1] = Y \rangle$\par
	$\quad i \geq 0 \land i < N - 1 \land AA[i] = X \land AA[i+1] = Y \land A[i] = Y \land A[i + 1] = X$\par
	$\Rightarrow \quad \langle p \land q \Rightarrow p \rangle$\par
	$\quad A[i] = Y \land A[i + 1] = X$\par
	$\blacksquare$\par
\end{absolutelynopagebreak}

\end{document}
